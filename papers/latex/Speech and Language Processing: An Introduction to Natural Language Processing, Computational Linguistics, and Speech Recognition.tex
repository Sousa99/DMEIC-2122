\documentclass{Paper_Summary}

% $REF{Speech and Language Processing: An Introduction to Natural Language Processing, Computational Linguistics, and Speech Recognition, Martin J. (2008)}
% $TITLE{Speech and Language Processing: An Introduction to Natural Language Processing, Computational Linguistics, and Speech Recognition}
% $AUTHOR{Martin J}
% $DATE{2008}

% $START-DATE{02/10/2021}
% $END-DATE{03/10/2021}

% ================================== VARIABLES ================================== 
\renewcommand{\varpapertitle}{{Speech and Language Processing: An Introduction to Natural Language Processing, Computational Linguistics, and Speech Recognition}}
\renewcommand{\varpaperauthor}{Martin J}
\renewcommand{\vardate}{{September 2021}}
% ================================== ========= ================================== 

\begin{document}
\makepapertitle

\breakline

\begin{center}
    \section*{Focus}
\end{center}
    
    Another paper that does not directly approach the problem of schizophrenic psychosis identification or signaling. The subject is simply Natural Language, both processing/analysis and generation/synthesis.

    The paper, or in this case manual, touches on the various paradigms existent in Natural Language in the 70s:
    \begin{itemize}
        \item \textbf{Stochastic}: This paradigm uses mainly \emph{Hidden Markov Models (HMM)} and the metaphors of the noisy channel and decoding. This paradigm deals with Natural Language in terms of probabilities, which can be analyzed and tested but cannot be precisely predicted.
        \item \textbf{Logic Based}: Use of logical systems such as \emph{Prolog} or \emph{Lexical Functional Grammar (LFG)} to analyze and process Natural Language. Although it refers to them as logical systems, their focus is to understand the interdependencies between POS components and structuring them logically. But there is another paradigm more focused on understanding the origin of Language to model it precisely.
        \item \textbf{Natural Language Understanding}: Focuses on the creation of natural language understanding systems with conceptual and theoretical knowledge of structures like plans, scripts, goals, and human-memory organization.
        \item \textbf{discourse modeling}: This paradigm started trying to develop programs capable of reference identification (identifying what do the references in sentences such as pronouns refer to).
    \end{itemize}

    There is a wide range of paradigms for Natural Language Processing and Generation. This wide range of paradigms is a consequence of the also wide range of uses and interests.

    Although the manual describes \emph{Logic Based} and \emph{Natural Language Understanding} as two different paradigms, it also points out their interconnectedness, later on even merging into a single unified paradigm.

    With the continuous developments of techniques, there was a unification of the various paradigms along with the integration with new and more promising approaches. Statistical analysis of Natural Language, use of finite-state automata, finite-state transducers, and Machine Learning techniques were all new techniques integrated into the field.

\breakline

\newpage

\section{Psychosis Characteristics}
    \emph{* None to discuss, not the objective of this paper *}

\section{Techniques}
    \emph{* None to discuss, not the objective of this paper *}

\section{Metrics}
    \emph{* None to discuss, not the objective of this paper *}

\section{Problems}
    \emph{* None to discuss, not the objective of this paper *}


\section{Final Remarks}
    \emph{* None to discuss, not the objective of this paper *}

\breakline

\begin{center}
    \section*{Possibly Useful Citations}
\end{center}

    \begin{itemize}
        % $CITATION{Understanding Language}
        \item \textbf{(Martin J, 2008, p. 14)} : "But language is not aeronautics. Cribbing from nature is sometimes useful for aeronautics (after all, airplanes do have wings), but it is particularly useful when we are trying to solve human-centered tasks."
    \end{itemize}

\end{document}
