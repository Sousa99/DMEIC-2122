\documentclass{Paper_Summary}

% $REF{Reduced Sensitivity to Linguistic Context in Schizophrenic Thought Disorder: Evidence From On-Line Monitoring for Words in Linguistically Anomalous Sentences, Kuperberg G. Mcguire P. David A. (1998), 423-434}
% $TITLE{Reduced Sensitivity to Linguistic Context in Schizophrenic Thought Disorder: Evidence From On-Line Monitoring for Words in Linguistically Anomalous Sentences}
% $AUTHOR{Kuperberg G. Mcguire P. David A.}
% $DATE{1998}

% $START-DATE{06/10/2021}
% $END-DATE{07/10/2021}

% ================================== VARIABLES ================================== 
\renewcommand{\varpapertitle}{{Reduced Sensitivity to Linguistic Context in Schizophrenic Thought Disorder: Evidence From On-Line Monitoring for Words in Linguistically Anomalous Sentences}}
\renewcommand{\varpaperauthor}{Kuperberg G. Mcguire P. David A.}
\renewcommand{\vardate}{{October 2021}}
% ================================== ========= ================================== 

\begin{document}
\makepapertitle

\breakline

\begin{center}
    \section*{Focus}
\end{center}
    
    This paper has an objective the expansion of the hypothesis that \emph{though disordered schizophrenics (TD)} have a reduced sensibility to linguistic deviations, unable to notice them.

    The paper starts by stating that although there have been previous studies that suggested that TD schizophrenic patients have a decreased sensitivity to these aberrations, however, others have defended an analogous sensitivity to a control group, and others even suggest that schizophrenic patients display more sensitivity to these deviations.

    This study defends the hypothesis that the studies that reached the last two conclusions either have not taken into account the difference between \emph{thought disordered schizophrenia} and \emph{non-thought disordered schizophrenia} or suffered from particular experimental conditions.
    
    The study comprised of three different tasks: \textbf{On-Line Word Monitoring Task}, \textbf{Off-Line Anomaly Detection Task} and a \textbf{Verbal Fluency Test}.

    In order to execute the first and second tasks, 32 nouns were selected. For each noun, a pair of sentences was created:
    \begin{itemize}
        \item The first sentence provided minimal context related to the target noun.
        \item The second sentence provided more context related to the target noun and followed a fixed structure.
    \end{itemize}

    To these, previously discussed, 32 pairs of phrases 44 were added. These 44 pairs of sentences had some pragmatical, semantic, or syntactically implausible error occurred. These errors were created by varying the verb preceding the target noun.

\breakline

\newpage

\section{Psychosis Characteristics}
    \begin{itemize}
        % $CHARACTHERISTIC{Absence of Self Monitoring; 1}
        \item \textbf{Failure to Self Monitor}: The authors defend due to the comparably high number of \emph{preservation} during the test of \emph{Verbal Fluency} that the thought disordered schizophrenic patients have some absense of the capability to self-monitor internal errors.
        % $CHARACTHERISTIC{Distractibility; 0.5}
        \item \textbf{Distractibility}: Associated with the disconnected component of formal thought.
        % $CHARACTHERISTIC{Poverty of Speech; 0.5}
        \item \textbf{Poverty of Speech}: Throughout the paper, the authors describe schizophrenic patients with thought disorder as with poverty of speech.
    \end{itemize}

\section{Techniques}
    \begin{itemize}
        % $TECHNIQUE{On-Line Word Monitoring Task; 1}
        \item \textbf{On-Line Word Monitoring Task}: The authors presented a card to the test subject with the target noun. This card was presented so that they would not forget the target noun. They would wear headphones, and every time they heard the target word being spoken should click a red button.
        % $TECHNIQUE{Off-Line Anomaly Detection Task; 1}
        \item \textbf{Off-Line Anomaly Detection Task}: The author would re-enumerate the previously spoken sentences, and for each, the test subject would state and discuss if there was any transgression.
        % $TECHNIQUE{Verbal Fluency Test; 1}
        \item \textbf{Verbal Fluency (FAS)}: The authors would ask the test subjects to enumerate, for 1 minute, as many words commencing with \emph{F}, then with \emph{A} and finally with \emph{S}.
    \end{itemize}

\section{Metrics}
    \begin{itemize}
        % $METRIC{Reaction Time}
        \item \textbf{Reaction}: During the \emph{On-Line Word Monitoring Task} reaction times were recorded for each subject and each sentence.
        % $METRIC{Reaction Time - Anticipations}
        \item \textbf{Reaction Anticipations}: During the \emph{On-Line Word Monitoring Task} anticipated clicks on the red button were recorded. If the test subject pressed the button and had a reaction time of less than 100ms (since this made it unlikely that the word was even processed).
        % $METRIC{Reaction Time - Misses}
        \item \textbf{Reaction Misses}: During the \emph{On-Line Word Monitoring Task} it was recorded if the subject did not press the button for a given sentence with the target noun.
        % $METRIC{Anomaly Detection Percentage}
        \item \textbf{Anomaly Detection Percentage}: During the \emph{Off-Line Anomaly Detection Task} it was recorded if the subject correctly classified the sentence and, in the case that it was given, the justification for the selection.
        % $METRIC{Verbal Fluency Score}
        \item \textbf{Verbal Fluency Score}: The Verbal Fluency Score was recorded.
        % $METRIC{Verbal Fluency Repetitions}
        \item \textbf{Verbal Fluency Repetitions}: During the Verbal Fluency Test, the authors recorded the number of repetitions of previously mentioned items.
        % $METRIC{Verbal Fluency Persevations}
        \item \textbf{Verbal Fluency Persevations}: During the Verbal Fluency Test, the authors recorded the number of items that were enumerated but belonged to a previous category.
        % $METRIC{Verbal Fluency Neologisms}
        \item \textbf{Verbal Fluency Neologisms}: During the Verbal Fluency Test, the authors recorded the number of non-words mentioned.
        % $METRIC{Verbal Fluency Associations}
        \item \textbf{Verbal Fluency Associations}: During the Verbal Fluency Test, the authors recorded the number of items stated that did not belong to the category or a previous one.
    \end{itemize}

\section{Problems}
    \begin{itemize}
        % $PROBLEM{Small Sample}
        \item \textbf{Small Sample}: Although not terrible the total numebr of test subjects was 47 divided into 3 main categories.
        % $PROBLEM{Restrictive Sample}
        \item \textbf{Female Test Subjects Restricted}: The proportion of female test subjects when compared to male is really small, which may cause some deviations in results analysis.
    \end{itemize}


\section{Final Remarks}

    The overall reaction time of the Thought Disorder Schizophrenics was higher than the Non-Thought Disordered Schizophrenic patients. These, in turn, had a higher reaction time than normal test subjects.

    Although TD schizophrenics had a higher reaction time than the rest of the groups, it did not increase as much as the remaining groups for each type of error in the sentences. This indicates some flexibility in terms of abnormal sentences, which confirms the initial hypothesis of the authors.

    The remainder of the results were as expected. TD schizophrenic patients displayed the highest error rate in the Off-Line Anomaly Detection task, which was directly correlated with the degree of thought disorder.
    TD schizophrenic patients also displayed the lowest verbal fluency score. The proportion of errors in this test was correlated with the degree of thought disorder but not the overall number of items produced.

\breakline

\begin{center}
    \section*{Possibly Useful Citations}
\end{center}

    \begin{itemize}
        % $CITATION{Schizophrenia Heterogenity}
        \item \textbf{(Kuperberg G. Mcguire P. David A., 1998, p. 432)} : "positive thought disorder is itself heterogeneous and is unlikely to be a unitary construct"
        % $CITATION{Schizophrenia as a State}
        \item \textbf{(Kuperberg G. Mcguire P. David A., 1998, p. 432)} : "suggesting on-line sensitivity to linguistic context is a state rather than a trait feature of schizophrenic thought disorder"
    \end{itemize}

\end{document}
