\documentclass{Paper_Summary}

% $REF{Speech Fluency and Schizophrenic Negative Signs, Alpert M, Kotsaftis A, Pouget E. Schizophrenia Bulletin, (1997), 171-177, 23(2)}
% $TITLE{Speech Fluency and Schizophrenic Negative Signs}
% $AUTHOR{Alpert M, Kotsaftis A, Pouget E.}
% $DATE{1997}

% $START-DATE{01/10/2021}
% $END-DATE{02/10/2021}

% ================================== VARIABLES ================================== 
\renewcommand{\varpapertitle}{{Speech Fluency and Schizophrenic Negative Signs}}
\renewcommand{\varpaperauthor}{Alpert M, Kotsaftis A, Pouget E.}
\renewcommand{\vardate}{{October 2021}}
% ================================== ========= ================================== 

\begin{document}
\makepapertitle

\breakline

\begin{center}
    \section*{Focus}
\end{center}
    
This paper studies the correlation between negative signs of schizophrenia, mainly flat effect and alogia through metrics such as pauses and hesitations.
It is important to distinguish which types of pauses were identified in this paper / how they were categorized:
\begin{itemize}
    \item \textbf{Swithching Pause}: Which occur at the beggining of the discourse right after the interviewer has finished. The time taken by the process of intreperting what was just stated by the interviewer or for preparatiion of the following discourse.
    \item \textbf{Pause between Clauses}: Longer pauses are associated with longer following clauses, since this take a bigger demand on cognitive processes. But there is no evidence to suggest that this type of pauses are correlated with the severity of alogia.
    \item \textbf{Pause within Clauses}: This type of pauses reflect an ill preparation by the speaker to mentally organize the discourse before its spoken or a difficulty to adapt to an ongoing discourse. Although previous studies had shown that these do not happen more frequently fro schizophrenic patients, instead they are simply longer. There are various reasons to why one could occur: \emph{speech repairs}, \emph{false starts} and \emph{repetitions}.
\end{itemize}

The papaer notes that for each subtype of pauses there is an adequate / normal reason why they should appear, and that they either represent normal processes that occur during the process of speech formulation or show empathy towards the listener, by either allowing that person to intrepert wwhat was jsut said or for 'dramatic' effects. Nontheless, when too extensive it shows a deficiency in the speech generation by the speaker.

\breakline

\newpage

\section{Psychosis Characteristics}
    \begin{itemize}
        % $CHARACTHERISTIC{Poverty of Speech; 1}
        \item \textbf{Flat Effect}: Charactherized by a lack of emotion and empathy through the carried out discourse, incapable of deriving emotions such as pleasure or upsetiness from it.
        % $CHARACTHERISTIC{Poverty of Speech; 1}
        \item \textbf{Alogia}: Discourse charactherized by extremelly short sentences, focusing on only strictly answering the question at hand.
    \end{itemize}

\section{Techniques}
    \begin{itemize}
        % $TECHNIQUE{Semi-Structured Interview; 1}
        \item \textbf{Semi-Structured Interview}: For the study a five to ten minute interview was carried out with the patient taking into account that the main purpose was to identify the pauses during the speech and were they happened and not particulary interested in the content / discourse.
        % $TECHNIQUE{Review of Altered Tapes; 1}
        \item \textbf{Review of Altered Tapes}: Three tapes were presented to clinicians. One of the tapes remained as orignal (from one of the pacient recordings), another was altered whith the time of each pause reduced to half and another where the time for each pause was doubled. The clinicians would then rate the tape for flatness and alogia.
    \end{itemize}

\section{Metrics}
    \begin{itemize}
        % $METRIC{Schedule for Affective Disorders and Schizophrenia}
        \item \textbf{Schedule for Affective Disorders and Schizophrenia}: Collection of Psychiatric diagnostic criteria composed of rating scales, this result is obtained by doing a semi-structured interview. The results are then displayed with a score from zero to three. — \href{https://en.wikipedia.org/wiki/Schedule_for_Affective_Disorders_and_Schizophrenia}{Further Information}
        % $METRIC{Silent Pauses Identified}
        \item \textbf{Silent Pauses Identified}: The timestamp and time duration of each \emph{in turn} pause was identified.
        % $METRIC{Response Latency}
        \item \textbf{Response Latency Identified}: The duration of the \emph{response latency} of the interviewee was measured.
        % $METRIC{Scale for the Assessment of Negative Symptoms}
        \item \textbf{Scale for the Assessment of Negative Symptoms}: Scale used to measure negative symptoms of schizophrenia. Each domain / subscale is measure from zero to five. — \href{https://en.wikipedia.org/wiki/Scale_for_the_Assessment_of_Negative_Symptoms}{Further Information}
        % $METRIC{DSM Criterion}
        \item \textbf{DSM-IV Criterion}: DSM-IV refers to a manual called \emph{Diagnostic and Statistical Manual of Mental Disorders} on its fourth edition. Its one of the best known publications, and contributted majorily into improving the reliability of psychiatric diagnosis. \href{https://en.wikipedia.org/wiki/Diagnostic_and_Statistical_Manual_of_Mental_Disorders#DSM-IV_(1994)}{Further Information}
    \end{itemize}

\section{Problems}
    \begin{itemize}
        % $PROBLEM{Accoustic Imprecision}
        \item \textbf{Accoustic Imprecision}: The paper indicates that better and more reliable results could have been achieved through the use of better rooms and techonological material.
    \end{itemize}


\section{Final Remarks}
    
    The paper is extemelly methodocial at trying to prove its points, not only by analysis of the data obtained but also by hypothesizing other explanations for the results obtained and justifying why / if they are unlikely.

    For example the paper finds as a possible justification for the flatness of response the possibly small patient lexicon, and then later refuses this theory since there appears to exist no correlation, justifying that a good estimator for the size of someone's lexicon is the ration of new vocabulary by the total number of items as well as IQ.

    Nontheless, the papers seems to conclude that there is a clear correlation between pausing, flat affect and alogia, and that all of them seem to have a clear relation to the popular rating scales used by clinicias, but that for example, pauses duration have little variation, something that naturally would be almost imperceptible to humans. Therefore, also concludes that clinicians seem to have an instricis and almost intinctly way of diagnosing patients even if using the propper scales and metrics.

\breakline

\begin{center}
    \section*{Possibly Useful Citations}
\end{center}
\emph{* None found *}

\end{document}
