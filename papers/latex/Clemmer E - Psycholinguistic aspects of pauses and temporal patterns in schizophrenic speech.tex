\documentclass{Paper_Summary}

% $REF: Psycholinguistic aspects of pauses and temporal patterns in schizophrenic speech, Clemer E. 1980
% $TITLE: Psycholinguistic aspects of pauses and temporal patterns in schizophrenic speech
% $AUTHOR: Clemmer, Edward J.
% $DATE: 1980

% $START-DATE: 21/09/2021
% $END-DATE: UNKNOWN

% ================================== VARIABLES ================================== 
\renewcommand{\varpapertitle}{{Psycholinguistic aspects of pauses and temporal patterns in schizophrenic speech}}
\renewcommand{\varpaperauthor}{Clemmer E.}
\renewcommand{\vardate}{{September 2021}}
% ================================== ========= ================================== 

\begin{document}
\makepapertitle

\breakline

\begin{center}
    \section*{Focus}
\end{center}
    
    The paper has it is implicit by the title takes a big focus on the pauses and some temporal metrics and patterns in the speech of schizophrenic patients.
    Although the methodology and the results obtained are interesting, the paper has big focus on its introduction where it discusses openly what defines a schizophrenic patient (and if it is possible at all), reasoning that since it is a possible diagnose people with schizophrenia it must exist (even if not explicitly defined) an commonality among these group of people.
    
    The paper starts by stating that even peculiarities that seem to be associated with schizophrenia seem to be ill-defined. In one side schizophrenia might lead to esoteric productions (compared to the ones of poets) but it might also show almost no language disturbances. To make things even more difficult schizophrenic speech is sometimes episodic meaning that sometimes these peculiarities transpire while at other times show almost no signs at all.

    Finally, the paper exposes that there are two possibilities for the origin of these peculiarities in speech:
    \begin{enumerate}
        \item Either a dysfunction of speech directly.
        \item More probably, a dysfunction of the cognitive capabilities and therefore causing these speech anomalies.
    \end{enumerate}

    Clemmer E. ends up defending that most agree that the speech anomalies are caused by cognitive deficits, but since these cognitive dysfunctions are hard to be analyzed most studies focus on the speech peculiarities taking into account that these might be originated by breaks on the cognitive flows of the patient.

\breakline

\newpage

\section{Schizophrenia Characteristics}
    \begin{itemize}
        % $CHARACTHERISTIC{Gibberish, 1}
        \item \textbf{Gibberish}: Unmeaningful sentences spoken as if it were meaningful.
        % $CHARACTHERISTIC{Absnormal Rhyming, 1}
        \item \textbf{Abnormal Rhyming}: Unexpected rhyming during normal discourse.
        % $CHARACTHERISTIC{Absence of Topic, 1}
        \item \textbf{Absence of Topic}: Discourse missing an internal coil to guide its reasoning or flow.
        % $CHARACTHERISTIC{Disruption of Syntax Rules, 1}
        \item \textbf{Disruption of the syntax rules}: Discourse does not make a correct assumption on what must be stated or can instead be omitted.
        % $CHARACTHERISTIC{Preoccupation with Syntax Rules, 1}
        \item \textbf{Preoccupation with too many syntax features}: Discourse continuously trying to be overcorrected by the speaker to an extreme.
        % $CHARACTHERISTIC{Absence of Self Monitoring, -1}
        \item \textbf{Failure to self monitor}: Speaker unable to self monitor itself and its discourse in terms of topic and syntax error. But disregarded since it is contested if this is a true characteristic since sometimes (and not that unusually) speakers are completely aware of themselves even more than controls.
        % $CHARACTHERISTIC{Disruption on Flow of Ideas, 2}
        \item \textbf{Disruption on flow of ideas}: Speaker is inconsistent with what is considered the natural flow of ideas and how each one typically originates from a connection with a previous one.
    \end{itemize}

\section{Techniques}
    \begin{itemize}
        % $TECHNIQUE{Matching of Control with Schizophrenic}
        \item \textbf{Matching of Control with Schizophrenic Patients}: Each schizophrenic patient was matched with a corresponding control. Matching was done according to the \underline{race} (matching exactly), \underline{sex} (matching exactly), \underline{age} (not matching exactly but the biggest difference being of two years) and \underline{years of educations} (matching exactly).
        % $TECHNIQUE{Reading}
        \item \textbf{Subjects Read a Story}: Each subject was asked to read two versions of one story previously chosen from a subset of two:
            \begin{itemize}
                \item Child Story.
                \item Dentist Story.
            \end{itemize}
        % $TECHNIQUE{Retelling}
        \item \textbf{Subjects Retold a Story}: Each subject had to retell the story they had just read before to the interviewee. This happened for both versions of the story individually and right after the reading of each.
        % $TECHNIQUE{Unusual Story}
        \item \textbf{Subject faced a Usual and Unusual Version of the Story}: Each subject was faced with an usual version of the story they had previously read (and retold). This version changed only in a small point that when read alone from the rest of the story was plausible but than when read in the context of the remainder of the story made it implausible or incorrect altogether.
    \end{itemize}

\section{Metrics}
    \begin{itemize}
        % $METRIC{Speech Duration}
        \item \textbf{Speech Duration}: The speech duration of each speech was recorded.
        % $METRIC{Speech Syllable Count}
        \item \textbf{Speech Syllable Count}: The number of syllables for each speech was recorded.
        % $METRIC{Silent Pauses Identified}
        \item \textbf{Silent Pauses were identified}: The location and duration of silent pauses in the subject speeches was identified.
        % $METRIC{Silent Pauses Classified}
        \item \textbf{Silent Pauses were classified}: The type of silent pauses was identified:
        \begin{itemize}
            \item \underline{B Type}: Pauses \underline{between} immediate constituent units.
            \item \underline{W Type}: Pauses \underline{within} immediate constituent units.
        \end{itemize}
        Classified according to Clark and Clark's (1977, p. 248) notion of "constituent plan". Taking into consideration that speakers plan than more than word at a time while speaking but not necessarily a whole sentence.
        % $METRIC{Filled Pauses Identified}
        \item \textbf{Filled Pauses were identified}: The number and location of filled pauses were identified (\emph{ah}, \emph{uh} and \emph{hm}).
        % $METRIC{Repeats Identified}
        \item \textbf{Repeats were identified}: Repetitions of single phonemes to extend stretches of 4 or 5 words without being explicitly necessarily (since they do not convey any extra information).
        % $METRIC{False Starts Identified}
        \item \textbf{False Starts were identified}: Incomplete utterances or corrected words during the subject's speech.
        % $METRIC{Parenthetical Remarks Identified}
        \item \textbf{Parenthetical Remarks were identified}: The number and location of filled pauses with conventional pause fillers (\emph{well}, \emph{you know}, and \emph{I guess}). 
    \end{itemize}

\section{Problems}
    \begin{itemize}
        % $PROBLEM{Restrictive Sample}
        \item \textbf{Only White Caucasian subjects}: Study was carried out exclusively with subjects of a white race.
        % $PROBLEM{Small Sample}
        \item \textbf{Rather small sample}: Study carried out with 40 subjects (being 20 the control subjects and 20 schizophrenic patients).
        % $PROBLEM{Schizophrenic under Medication}
        \item \textbf{Patients under medication}: Study carried out exclusively with patients taking medication (phenothiazines) which might cause speech disturbances meaning that the classification might be focusing on the effects of this medication instead of the diagnosis.
        % $PROBLEM{Schizophrenic under Lobotomy}
        \item \textbf{Patients had Lobotomy}: Study carried with two patients in which was done a lobotomy which could have worsened speech impairments and therefore affected results.
        % $PROBLEM{Difference with other Neurological Disorder}
        \item \textbf{Does not cover other Psychoses or Neurological Disorders}: Even the author admits that this study does not allow for the studying of how other psychoses and neurological disorders differs from schizophrenia and if in fact the results obtained describe schizophrenia or in more general manner these types of disorders.
    \end{itemize}


\section{Final Remarks}

\breakline

\begin{center}
    \section*{Possibly Useful Citations}
\end{center}

    \begin{itemize}
        % $CITATION{Classifying Schizophrenic}
        \item \textbf{(Brown, 1973, p. 396)} : "must have some common quality or qualities, or else why are they labeled with one term?"
    \end{itemize}

\end{document}
