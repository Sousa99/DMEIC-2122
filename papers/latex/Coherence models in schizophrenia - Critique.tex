\documentclass{Paper_Summary}

% ================================== VARIABLES ================================== 
\renewcommand{\varpapertitle}{{Coherence models in schizophrenia}}
\renewcommand{\varpaperauthor}{Just S. Haegert E. KpYánová N. et al.}
\renewcommand{\vardate}{{November 2021}}
% ================================== ========= ================================== 

\begin{document}
\makepapertitle

\breakline

\begin{center}
    \section*{Critique}
\end{center}

    The authors focused on the correct classification/differentiation of three groups: controls, schizophrenic patients with thought disorder, and schizophrenic patients without thought disorder. The entire methodology comes as a continuation from a previous paper, which is beneficial since the first had a rather small sample size, which gave less confidence to the results obtained.

    For this reason, the paper does almost the same as the previous one, and there is not much of a critique available to it since it would be essentially a repetition of the previous one. Of note that this study had mainly two differences from the initial one:
    \begin{itemize}
        \item The authors were responsible for the transcription process and identification of ambiguous pronouns. The authors defend this choice by stating that the proposed \emph{referential coherence model} is prone to error but never supporting this affirmation with a clear justification. On the other hand, the authors knowing to which group is each recording associated with possibly add bias to the dataset.
        \item The sample size of the study had improved when compared to the initial study, so at least now the conclusions achieved have a higher confidence level.
    \end{itemize}

\breakline

\end{document}
