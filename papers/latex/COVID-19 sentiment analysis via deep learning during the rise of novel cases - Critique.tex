\documentclass{Paper_Summary}

% ================================== VARIABLES ================================== 
\renewcommand{\varpapertitle}{{COVID-19 sentiment analysis via deep learning during the rise of novel cases}}
\renewcommand{\varpaperauthor}{Chandra R. Krishna A.}
\renewcommand{\vardate}{{October 2021}}
% ================================== ========= ================================== 

\begin{document}
\makepapertitle

\breakline

\begin{center}
    \section*{Critique}
\end{center}

    This paper focused on the \textbf{sentiment analysis} of \emph{COVID-19} related texts, in this case \emph{tweets}. Important to note that the main subject of this paper has almost nothing to do with the thesis' future work.

    The authors focused on the analysis and the temporal development of these sentiments throughout the pandemic. The authors also tried to analyze how the sentiment correlated with the number of cases and their development. The future thesis' work could focus instead on the sentiment progression throughout the description of \textbf{affective images} since this is the only test where the subject is freer to express itself and its world view.

    To this purpose, the thesis' work would involve the segmentation of discourse into utterances. The use of the referred variations of \textbf{Recurrent Neural Networks} could provide a compelling and accurate methodology to classify the sentiment of each utterance.

    For this methodology, the thesis would need an annotated corpus for the model to learn what is associated with a positive or negative sentiment and then extrapolate to the patient's recordings. One such corpus is publicly available on \emph{Kaggle} however it is only annotated with positive and negative sentiment.

    This paper served the purpose of a first introduction into the subject of \textbf{sentiment analysis} particularly with \textbf{recursive neural networks}.

    \vspace{-5mm}

\breakline

\end{document}
