\documentclass{Paper_Summary}

% ================================== VARIABLES ================================== 
\renewcommand{\varpapertitle}{{An automated method to analyze language use in patients with schizophrenia and their first-degree relatives}}
\renewcommand{\varpaperauthor}{Elvevåg B. Foltz P. Rosenstein M. et al.}
\renewcommand{\vardate}{{November 2021}}
% ================================== ========= ================================== 

\begin{document}
\makepapertitle

\breakline

\begin{center}
    \section*{Critique}
\end{center}

    This paper was initially selected with the objective of better understanding the heritability of schizophrenia leading to psychosis. Either through the \emph{introduction}, \emph{methodology} or even \emph{conclusions} the hope was to find relevant information that would either support the heritability theory or give evidence towards this theory's refusal.

    This paper, in its \emph{introduction}, touches on the subject of this theory but uses it as a \emph{known fact} instead of trying to justify it. This use comes as a surprise since a more recent study stated that there had been clear support of this theory but only with \textbf{50\% concordance} among \emph{monozygotic twins} instead of the expected \textbf{100\% concordance}.

    Apart from this problem, the study follows the standard approach from papers read at the same time. There is a big focus on the recent technique, at least at the time, of \emph{Latent Semantic Analysis} and in providing support for its use against the, until then more common, discourse and syntactic measures.

    Similar to the previous study, the authors gave a somewhat in-depth description of the techniques used but did not provide an in-depth description of the features used for each classificator and their meaning when related to the problem, generalizing only to their type.

\breakline

\end{document}
