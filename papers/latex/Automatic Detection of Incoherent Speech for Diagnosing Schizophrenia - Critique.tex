\documentclass{Paper_Summary}

% ================================== VARIABLES ================================== 
\renewcommand{\varpapertitle}{{Automatic Detection of Incoherent Speech for Diagnosing Schizophrenia}}
\renewcommand{\varpaperauthor}{Iter D. Yoon J. Jurafsky D.}
\renewcommand{\vardate}{{November 2021}}
% ================================== ========= ================================== 

\begin{document}
\makepapertitle

\breakline

\begin{center}
    \section*{Critique}
\end{center}

    The paper here referred served mainly for two factors:
    \begin{itemize}
        \item In order to properly study any \emph{Natural Language Domain} the choice of the correct pre-processing techniques is essential. This fact is especially important in domains where acquiring such data can become cumbersome.
        \item This study was the first to shift focus into the \emph{referential coherence} analysis. Although previous papers had mentioned such characteristics in schizophrenic patients, none had taken advantage. When literature refers to schizophrenic patients' incoherent discourse, in fact, what is noticeable is that for us, the meaning of the sentence is not apparent, but that with an extra effort, one could estimate how the sentence came to be generated, or at least the originating thoughts that created this sentence. This inability to properly convey a message is, sometimes, highlighted by the ambiguous reference to entities.
    \end{itemize}

    Although the previous two points and the fact that the authors expanded on a standard technique already used for so long, the conclusions should be taken into consideration carefully. The study concluded with little confidence in their results since it has almost no generalization power, mainly for two factors:
    \begin{itemize}
        \item The sample size is rather small, even if compared to the usual sample sizes in studies on this domain.
        \item The subjects had a rather small distribution over the entire population. The majority of the subjects were male and had a small range of ages.
    \end{itemize}

\breakline

\end{document}
