\documentclass{Paper_Summary}

% ================================== VARIABLES ================================== 
\renewcommand{\varpapertitle}{{Automated coherence measures fail to index thought disorder in individuals at risk for psychosis}}
\renewcommand{\varpaperauthor}{Hitczenko K. Cowan H. Mittal V. et al.}
\renewcommand{\vardate}{{November 2021}}
% ================================== ========= ================================== 

\begin{document}
\makepapertitle

\breakline

\begin{center}
    \section*{Critique}
\end{center}

    Although an immediate conclusion would suggest that content analysis from the discourse of patients and controls only relates to the social and education level differences amongst these two groups, several studies have suggested otherwise, even some that take into account distribution over race and educational level.

    Such concerns need to be considered since they may propagate harmful societal biases. Instead of helping certain societal groups in their diagnosis and stigmas might propagate other societal groups' stigmas.

    The problems referred by the authors when it comes to biases of educational level in discourse in techniques such as \emph{Latent Semantic Analysis}. \emph{LSA} is biased to repetitions and lengthy sentences. The latter might reflect educational differences.

    The future thesis' work should consider these differences in population groups to avoid propagating societal stigmas. This analysis can be achieved through several different approaches:
    \begin{itemize}
        \item Control control and patient's differences, such as educational and racial characteristics, assuring that they are similar or at least similar enough for the execution of the study.
        \item Once chosen, the features for classification must be analyzed for possible correlations with environmental and societal factors. If some correlation is identified, then their possible impact on the results should be openly discussed.
    \end{itemize}

\breakline

\end{document}
