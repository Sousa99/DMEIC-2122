\documentclass{Paper_Summary}

% ================================== VARIABLES ================================== 
\renewcommand{\varpapertitle}{{Reduced Sensitivity to Linguistic Context in Schizophrenic Thought Disorder: Evidence From On-Line Monitoring for Words in Linguistically Anomalous Sentences}}
\renewcommand{\varpaperauthor}{Kuperberg G. Mcguire P. David A.}
\renewcommand{\vardate}{{October 2021}}
% ================================== ========= ================================== 

\begin{document}
\makepapertitle

\breakline

\begin{center}
    \section*{Critique}
\end{center}

    Although most of the data during the thesis has already been procured, the test in which subjects enumerate animals that start with a given letter can correspond to the Verbal Fluency task. With this approach, it will be important to measure both \emph{neologisms}, \emph{repetitions}, \emph{preservations} and \emph{associations}.

    The way the paper defends its refutations is efficient. During the thesis, it will be critical to think of this process.

    Not much else needs to be cited from this paper. It focuses mainly on temporal data analysis, verbal fluency, and the detection of sentence anomalies. Although the paper approaches the subject of \emph{context} or \emph{pragmatics}, it is with a direct test method. During the thesis, the context will be analyzed mainly through ordinary discourse taking advantage of the relationship between spoken words instead of being tested directly.

\breakline

\end{document}
