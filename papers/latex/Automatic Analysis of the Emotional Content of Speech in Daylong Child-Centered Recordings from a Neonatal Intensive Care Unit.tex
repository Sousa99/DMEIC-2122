\documentclass{Paper_Summary}

% $REF{Automatic Analysis of the Emotional Content of Speech in Daylong Child-Centered Recordings from a Neonatal Intensive Care Unit Vaaras E. Ahlqvist-Björkroth S. Drossos K et al. (2021), 3380-3384}
% $TITLE{Automatic Analysis of the Emotional Content of Speech in Daylong Child-Centered Recordings from a Neonatal Intensive Care Unit}
% $AUTHOR{Vaaras E. Ahlqvist-Björkroth S. Drossos K et al}
% $DATE{2021}

% $START-DATE{23/10/2021}
% $END-DATE{23/10/2021}

% ================================== VARIABLES ================================== 
\renewcommand{\varpapertitle}{{Automatic Analysis of the Emotional Content of Speech in Daylong Child-Centered Recordings from a Neonatal Intensive Care Unit}}
\renewcommand{\varpaperauthor}{Vaaras E. Ahlqvist-Björkroth S. Drossos K et al}
\renewcommand{\vardate}{{October 2021}}
% ================================== ========= ================================== 

\begin{document}
\makepapertitle

\breakline

\begin{center}
    \section*{Focus}
\end{center}

    This paper focused on identifying a two-dimensional emotion on pre-recorded interactions from \emph{Neonatal Intensive Care Units (NICUs)}. The \emph{Auditory environment by Parents of Preterm infant; Language development and Eye-Movements (APPLE)} gathered these recordings.

    The study wanted to evaluate if non-labeled corpora could be used in order to study a given problem. The problem in focus in this study is whether the emotional \textbf{valence} and \textbf{arousal} could have and justify long-term effects on children.

    Annotating the whole corpora was not an option for the author due to its size. The data was comprised of \emph{43 families} recorded for \emph{16 hours} which in total meant \emph{688 hours} of audio to be analyzed. Even after a careful selection of eight families for the \emph{train set} and four families for the \emph{test set} the data was too big to be manually labeled.

    These recordings captured all conversation in said rooms, which justified the need for a pre-processing stage where the speakers and their distance (as a feature for later analysis) was identified. The authors then eliminated nurses, doctors, and other children (who were not the key children) from the study corpus.

    Due to the nature and focus of this study not aligning in a direct way with the one from the future thesis, check the \emph{Critique} in order to understand better the purpose of this paper for the thesis.

    \vspace{-5mm}
\breakline

\newpage

\section{Psychosis Characteristics}
\emph{* None to discuss, not the objective of the paper *}

\section{Techniques}
\emph{* None to discuss, not the objective of the paper *}

\section{Metrics}
\emph{* None to discuss, not the objective of the paper *}

\section{Problems}
\emph{* None to discuss, not the objective of the paper *}


\section{Final Remarks}
\emph{* None to discuss, not the objective of the paper *}

\breakline

\begin{center}
    \section*{Possibly Useful Citations}
\end{center}
\emph{* None found *}

\end{document}
