\documentclass{Paper_Summary}

% $REF{Automatic Detection of Incoherent Speech for Diagnosing Schizophrenia, Iter D. Yoon J. Jurafsky D. (2018), 136-146}
% $TITLE{Automatic Detection of Incoherent Speech for Diagnosing Schizophrenia}
% $AUTHOR{Iter D. Yoon J. Jurafsky D.}
% $DATE{2018}

% $START-DATE{09/11/2021}
% $END-DATE{09/11/2021}

% ================================== VARIABLES ================================== 
\renewcommand{\varpapertitle}{{Automatic Detection of Incoherent Speech for Diagnosing Schizophrenia}}
\renewcommand{\varpaperauthor}{Iter D. Yoon J. Jurafsky D.}
\renewcommand{\vardate}{{November 2021}}
% ================================== ========= ================================== 

\begin{document}
\makepapertitle

\breakline

\begin{center}
    \section*{Focus}
\end{center}

    The presented study focused on the \textbf{coherence analysis} of transcribed interviews with both schizophrenic patients and controls. Through this coherence analysis, the authors improved on already defined methodologies and techniques. With these improvements, the authors hoped to achieve better results on the classification of controls and schizophrenic/schizoaffective disordered patients.

    The original methodology of \emph{Latent Semantic Analysis} relies on the assumption that coherence can be measured by the \emph{cossine similarity} of segments of tokens (or \emph{bags of words}). Due to the very nature of the methodology:
    \begin{itemize}
        \item \textbf{Repetitions} of certain words or the constant use of synonyms generates a higher coherence metric, which is particularly harmful in the current domain since \emph{word reptition} has been identified in some literature as a possible schizophrenia's symptom. An algorithm should then:
        \begin{itemize}
            \item Be impervious to such factors as word repetitions, which could also be detrimental (although not so much) since it is a possible symptom of schizophrenia.
            \item Distinguish between \emph{abnormal} and \emph{normal} word repetition.
        \end{itemize} 
        \item \textbf{Sentence length} is correlated with a higher coherence since it bags more words together, which in turn increases the likelihood of two words being semantically related and therefore increases (artificially) the sentences' coherence.
    \end{itemize}

    The authors even differentiate between two different models for the measurement of coherence that rely both on \emph{Latent Semantic Analysis}:
    \begin{itemize}
        \item \textbf{Tangentiality Model}: Latent Semantic Analysis is applied to the pair question-answer of the interviews. Important to note that this does not mean that the unified answer must be taken as a whole, this one can be segmented, and each segment is paired with the aforementioned question. The coherence allows for the analysis of how distant is the answer to the question.
        \item \textbf{Incoherence Model}: Latent Semantic Analysis is applied to every pair of sentences or bag of words. In literature, it is common to use the minimum coherence on the document to express the overall coherence.
    \end{itemize}
    The authors suggested the above nomenclatures, so they must be taken carefully. 

    The improvements suggested by the authors against the standard \emph{Latent Semantic Analysis} mainly rely on:
    \begin{itemize}
        \item Pre-processing techniques, such as removal of conversational characteristics.
        \item Conversion from word embeddings (generated by tools such as \emph{Word2Vec}) to \emph{Sent2Vec}, which associated a embedding to a given sentence.
        \item Techniques such as: \emph{TD-IDF}, \emph{Smooth Inverse Frequency (SIF)} and \emph{Singular Value Decomposition (SVD)}.
    \end{itemize}

\breakline

\newpage

\section{Psychosis Characteristics}
    \begin{itemize}
        % $CHARACTHERISTIC{Disruption on Flow of Ideas; 1}
        % $CHARACTHERISTIC{Distractibility; 1}
        % $CHARACTHERISTIC{Loss of Coherence; 1}
        \item \textbf{Loss of Coherence}: Loss of coherence is the basis for the study, and it is seen as the main symptom from \textbf{Formal Thought Disorder (FTD)}. Other extremelly related symptoms includes basic \emph{disruption of the flow of ideas} and general \emph{distractibility}.
        % $CHARACTHERISTIC{Loss of Referential Standarts; 1}
        \item \textbf{Loss of Referential Standarts}: The second basis for the study is the referential problems displayed by schizophrenic patients. Is common for schizophrenic patients for:
        \begin{itemize}
            \item Use referential pronouns and never specifying the entity refered by the pronoun.
            \item The entity is referenced only after the pronoun is used (\emph{cataphora}), which although not impossible by controls, is less usual.
        \end{itemize}
    \end{itemize}

\section{Techniques}
    \begin{itemize}
        % $TECHNIQUE{Semi-Structured Interview; 1}
        \item \textbf{Semi-Structured Interview}: The authors ran interviews with each one of the subjects. Asking them 8 to 10 questions each for a specified amount of time. These interviews were recorded, transcribed, and tagged as spoken by the interviewer (trained clinician) or by the subject.
        % $TECHNIQUE{Latent Semantic Analysis; 1}
        \item \textbf{Latent Semantic Analysis (LSA)}: Latent Semantic Analysis provides an efficient technique to define a word's meaning. The meaning of each vector is expressed through a vector composed of N components instead of explicitly stated. The cosine of this vector with any other will express how similar the meaning of the two words is. The co-occurrences of words with other words determine the vector and its components.
        % $TECHNIQUE{Referential Coherence Model; 1}
        \item \textbf{Referential Coherence Model}: The authors used a referential coherence model with the objective of identifying document references to a single entity. The authors used a pre-trained co-reference model from another study. The transcribed interviews were fed to the model, which then returned lists of references to a single entity.
    \end{itemize}

\section{Metrics}
    \begin{itemize}
        % $METRIC{First Order Sentence Coherence}
        \item \textbf{First Order Sentence Coherence (FOC)}: After averaging out the \emph{LSA vectors} associated with the words present in each sentence, the study calculated the \emph{FOC} using the cosine of the average vector of each sentence and the subsequent sentence. This coherence was calculated with both models, baseline and each one of the improved versions.
        % $METRIC{Number of Ambiguous References}
        \item \textbf{Number of Ambiguous Pronouns}: The number of ambiguous pronouns were used as a feature for later analysis. A reference was considered ambiguous if and only if the first element of a list, resulting of the \emph{Referential Coherence Model} was a \emph{pronoun}.
    \end{itemize}

\section{Problems}
    \begin{itemize}
        % $PROBLEM{Small Sample}
        \item \textbf{Small Sample}: The test sample was rather small, even when compared with the usual sample size of studies on this domain, which is already small. The small sample size could have impacted greatly the results achieved, decreasing the confidence in the results obtained.
        % $PROBLEM{Patients under Medication}
        \item \textbf{Patients under Medication}: The authors do not refer to whether the test subjects who were part of the schizophrenic group were under any medication. Medication can alter the characteristics and degree of severity that are associated with schizophrenic patients.
        % $PROBLEM{Restrictive Sample}
        \item \textbf{Restrictive Sample}: The sample had high inconsistencies in terms of their demographic characteristics. The majority of the test subjects were male, so the results may not be generalized to a bigger domain.
    \end{itemize}


\section{Final Remarks}

    The results obtained serve as a proof of concept for possible improvements against the classic and generalized implementation of \emph{Latent Semantic Analysis} by better targeting the concrete domain. However, note that due to the problems mentioned above, results might not be generalizable.

\breakline

\begin{center}
    \section*{Possibly Useful Citations}
\end{center}
\emph{* None found *}

\end{document}
