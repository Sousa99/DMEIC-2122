\documentclass{Paper_Summary}

% ================================== VARIABLES ================================== 
\renewcommand{\varpapertitle}{{A machine learning approach to predicting psychosis using semantic density and latent content analysis}}
\renewcommand{\varpaperauthor}{Rezaii N. Walker E. Wolff P.}
\renewcommand{\vardate}{{October 2021}}
% ================================== ========= ================================== 

\begin{document}
\makepapertitle

\breakline

\begin{center}
    \section*{Critique}
\end{center}
    
    This study resembles the methodology of the previous two since it tries to analyze the patient's content to grasp whether it displays characteristics evident of future onset psychosis of Clinical-High risk. This study also closely resembles the future thesis' work, except that it will instead focus on helping clinicians' evaluation of psychosis.

    However, there are some shortcomings with this paper. For one, it has an extraordinarily small sample which might be responsible for the high accuracies caused by overfitting.

    Nonetheless, this paper explored new techniques, which provides the thesis with more techniques to be used and explored. Besides, it also made it apparent that exploring new techniques and inventive solutions can be an option.

\breakline

\end{document}
