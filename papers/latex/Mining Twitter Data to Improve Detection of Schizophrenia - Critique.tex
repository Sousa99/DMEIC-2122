\documentclass{Paper_Summary}

% ================================== VARIABLES ================================== 
\renewcommand{\varpapertitle}{{Mining Twitter Data to Improve Detection of Schizophrenia}}
\renewcommand{\varpaperauthor}{McManus K. Mallory E. Goldfeder R. et al.}
\renewcommand{\vardate}{{October 2021}}
% ================================== ========= ================================== 

\begin{document}
\makepapertitle

\breakline

\begin{center}
    \section*{Critique}
\end{center}

    This paper analyzed public posts in a microblogging platform, in this case, Twitter, to classify a user as a schizophrenic patient from a control subject.

    Although alluring that such informal discourse can be used to classify users, the authors could have chosen better features and labels that relied less on a perspective of self-diagnosis.

    One of the features used relies on the number of words related to schizophrenia, which could be biased towards self-diagnosis or even simple curiosity towards the theme.
    For the training set, the labeled users were considered based on if they follow a community, which again is biased towards self-diagnosis or curiosity for the theme.

    Another problem with the study is that they do not justify the choice of features based on schizophrenic characteristics, which further weakens their choice.

    Finally, paramount to note that this paper was chosen to get a better grasp of sentiment analysis, but in reality, relies very little, if at all, on this subject.

\breakline

\end{document}
