\documentclass{Paper_Summary}

% $REF{Garmezy N, Process and Reactive Schizophrenia: Some Conceptions and Issues, Schizophrenia Bulletin, Volume 1, Issue 2, Fall 1970, Pages 30–74, https://doi.org/10.1093/schbul/1.2.30}
% $TITLE{Process and Reactive Schizophrenia: Some Conceptions and Issues}
% $AUTHOR{Garmezy N}
% $DATE{Fall 1970}

% $START-DATE{27/09/2021}
% $END-DATE{31/12/9999}

% ================================== VARIABLES ================================== 
\renewcommand{\varpapertitle}{{Process and Reactive Schizophrenia: Some Conceptions and Issues}}
\renewcommand{\varpaperauthor}{Garmezy N}
\renewcommand{\vardate}{{September 2021}}
% ================================== ========= ================================== 

\begin{document}
\makepapertitle

\breakline

\begin{center}
    \section*{Focus}
\end{center}
    This paper was read from a different perspective from the ones before and from the ones I expect to be reading in the future. I say this because this paper does not focus so much on the correct identification / classification of schizophrenia from a different group (being this other mental illnesses or controls) but instead focuses on the discussion of how schizophrenia is currently diagnosed and how has it been diagnosed since the beginning.

\breakline

\newpage

\section{Schizophrenia Characteristics}
\emph{* None to discuss, not the objective of this paper *}

\section{Techniques}
\emph{* None to discuss, not the objective of this paper *}

\section{Metrics}

\underline{Diagnosis by Prognosis} by Emil Kraepelin

    The paper starts by explaining that in the beginning the diagnosis was made solely in consequence of the prognosis. Meaning that \emph{exogenours} mental disorders were generally deemed recoverable while \emph{endogenous} were deemed as almost unrecoverable.
    These two possible diagnosis were attributed by comparison, with other cases that had already reached prognosis and by the history that could be retrieved of the patient, and deemed only correct or incorrect at prognosis. In the eventuality that the diagnosis was wrong (when prognosis was reached) it was associated as wrongful diagnosis and the characteristics and signs were never reconsidered.

    To add to the problem of this lack of reconsideration for the rethinking of metrics and signs of schizophrenia the very definition of \emph{recovery} was fragile and not consensual from investigator to investigator.

    Most of the traits used for describing schizophrenia are traits that nowadays would simply be associated with shut-in personality.

    This type of diagnosis was 'overthrown' due to various psychological progresses made, such as:
    \begin{itemize}
        \item \underline{Adolf Meyer} proposed that the fluctuations seen in mental illnesses such as schizophrenia come in waves, many times appearing by mundane problems and in time the mental state of the patient tends to deteriorate. Since there is a process of deterioration this would imply that there is a possible prevention of this deterioration and that this prevention is better achieved if the patients are not deemed recoverable / unrecoverable and their state is not blindly attributed to hereditary or endogenous factors.
        \item \underline{Eugen Bleuler}, which focused on various mental illnesses, noted that schizophrenia could sometimes take rather irregular and uncommon courses. Schizophrenia could be charactherized by rapid advancements, sudden halts, remissions, etc. By introducing this new view Bleuer opened the door for future studies to study possible signs of schizophrenia before the \emph{Onset} phase.
        \item \underline{Harry  Stack  Sullivan} which had several contributtions to the diagnosis of schizophrenia but I will touch only in a few. The main changes introduced by Sullivan were:
        \begin{enumerate}
            \item Increased focus on patient antecedents and the role that stress and challenges encountered had in its life.
            \item Role of sexual and social adaption needed to ameliorate schizophrenia and its symptoms.
            \item Rejection of binary polarizing and absolute sepparation between \emph{endogenous} and \emph{exogenous} caused schizophrenia.
            \item Reinforced the importante of medical intervention in order to favor the recovery of the patient.
        \end{enumerate}
    \end{itemize}

    All this progress helped in formulating what is now known as the \emph{Process / Reactive} dogma of schizophrenia and the sepparation from the dimension \emph{Favorable / Unfavorable}, although not completely uncorrelated.

\underline{Elgin  Prognostic  Scale} by Phyllis Wittman

    The Elgin Prognostic Scale (EPS) was made up of 20 subscales (after it was trimmed from one first original version composed of 30). Each scale was composed of a descriptive label and an weight that was attributed negative or positive, if the factor was deemed favorable or unfavorable respectively.

    Altough some initial studies showed that the scale was successfull at determining the prognosis of the disease from historical records it behaved only sligtly better than chance and almost just as good as the classification according only to the marital status of the patient, with almost half of the subscales being deemed as insignificant to the classification.

    There were various problems with this scale:
    \begin{enumerate}
        \item The label for the subscales although descriptive were imprecise and ambiguous,
        \item The characteristics / factors stated in the labels "patient is X or Y" sometimes did not refer to necessarily opposite terms.
        \item The worst problem, this scale relied greatly on extensive patient previous life-history which in most cases was not able of being obtained.
    \end{enumerate}

\underline{The Phillips Scale of Premorbid Adjustment in Schizophrenia}

    The main difference in regard to this scale in comparisson to the previous ones is that it relies on the improvement / worsening of the patient against a previous analysis according to 5 subscales:
    \begin{itemize}
        \item Recent Sexual Adjustment
        \item Social Aspects of Sexual Life during Adolescence and Immediately Beyond
        \item Social Aspects of Recent Sexual Life (differentited for under or above the age of 30)
        \item History or Personal Relations
        \item Recent Adjustment in Personal Relations
    \end{itemize}

    Each subscale was then scored from 0 to 5 or 1 to 6, being 3 always the fulcral point. If the score was higher it described unimprevements, if the value was lower describing improvements.

    Naturally the total score would go from 0 to 30, with scores from 0 to 15 describing \emph{good premorbid schizophrenic cases} and 16 to 30 denoting \emph{poor premorbid cases}.

    There were three advantages to this scale:
    \begin{enumerate}
        \item Avoided bias or ambiguous terms, relying on more concrete scales and descriptions for each value of each subscale.
        \item Demanded only a small ammount of patient life-history since it relied on improvement versus a baseline instead of absolute values.
        \item The scale has been extensively used, and its merit supported through various studies, both interdependent and independent.
    \end{enumerate}

    It also has its own problems when it comes to inadequaly defined variables, assignment of arbitrary weights, and others. But showed major improvements versus the previously defined metrics for diagnosis.

\emph{* To be concluded *}

\section{Problems}
\emph{* None to discuss, not the objective of this paper *}


\section{Final Remarks}
\emph{* To be concluded *}

\breakline

\begin{center}
    \section*{Possibly Useful Citations}
\end{center}

    \begin{itemize}
        % $CITATION{Diagnosis by Prognosis}
        \item \textbf{(Zilboorg  G.  and   Henry G.W., 1941)} : One cannot say that because a disease ends in a certain definite way it is a certain definite disease.
    \end{itemize}

\end{document}
