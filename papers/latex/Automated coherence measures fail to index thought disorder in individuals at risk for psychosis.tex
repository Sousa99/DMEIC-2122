\documentclass{Paper_Summary}

% $REF{Automated coherence measures fail to index thought disorder in individuals at risk for psychosis, Hitczenko K. Cowan H. Mittal V. et al. (2021), 129-150}
% $TITLE{Automated coherence measures fail to index thought disorder in individuals at risk for psychosis}
% $AUTHOR{Hitczenko K. Cowan H. Mittal V. et al.}
% $DATE{2021}

% $START-DATE{11/11/2021}
% $END-DATE{14/11/2021}

% ================================== VARIABLES ================================== 
\renewcommand{\varpapertitle}{{Automated coherence measures fail to index thought disorder in individuals at risk for psychosis}}
\renewcommand{\varpaperauthor}{Hitczenko K. Cowan H. Mittal V. et al.}
\renewcommand{\vardate}{{November 2021}}
% ================================== ========= ================================== 

\begin{document}
\makepapertitle

\breakline

\begin{center}
    \section*{Focus}
\end{center}

    The paper focused on the analysis for possible correlations between coherence measures studied in other papers and economic and societal factors.
    The entire paper revolves around the hypothesis that the recently achieved promising results in coherence analysis of the discourse of patients might only relate to differences in population characteristics.

    Such a hypothesis must always be considered in these studies since their inconsideration might lead to propagation or worsening of societal stigmas.

\breakline

\newpage

\section{Psychosis Characteristics}
\emph{* None to discuss, not the objective of this paper *}


\section{Techniques}
\emph{* None to discuss, not the objective of this paper *}


\section{Metrics}
\emph{* None to discuss, not the objective of this paper *}


\section{Problems}
\emph{* None to discuss, not the objective of this paper *}


\section{Final Remarks}

    The authors concluded through different strategies that the social factors are indeed correlated with the results obtained through coherence techniques and not correlated with the scores from current clinical diagnosis scales.

    These results seem to compromise current studies in coherence analysis of patients and controls, but such influence needs to be analyzed from corpus to corpus.

\breakline

\begin{center}
    \section*{Possibly Useful Citations}
\end{center}
\emph{* None found *}


\end{document}
