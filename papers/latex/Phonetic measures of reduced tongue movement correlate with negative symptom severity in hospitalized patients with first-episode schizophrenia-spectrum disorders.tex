\documentclass{Paper_Summary}

% $REF{Phonetic measures of reduced tongue movement correlate with negative symptom severity in hospitalized patients with first-episode schizophrenia-spectrum disorders, Covington M. Lunden S. Cristofaro S. et al. Schizophrenia Research, (2012), 93-95, 142(1-3)}
% $TITLE{Phonetic measures of reduced tongue movement correlate with negative symptom severity in hospitalized patients with first-episode schizophrenia-spectrum disorders}
% $AUTHOR{Covington M. Lunden S. Cristofaro S. et al.}
% $DATE{2012}

% $START-DATE{19/10/2021}
% $END-DATE{19/10/2021}

% ================================== VARIABLES ================================== 
\renewcommand{\varpapertitle}{{Phonetic measures of reduced tongue movement correlate with negative symptom severity in hospitalized patients with first-episode schizophrenia-spectrum disorders}}
\renewcommand{\varpaperauthor}{Covington M. Lunden S. Cristofaro S. et al.}
\renewcommand{\vardate}{{October 2021}}
% ================================== ========= ================================== 

\begin{document}
\makepapertitle

\breakline

\begin{center}
    \section*{Focus}
\end{center}
    
    This small study focuses on identifying schizophrenic patients by analyzing a possible correlation between phonetic measures of schizophrenic patients and the negative symptoms classification of the patient.

\breakline

\newpage

\section{Psychosis Characteristics}
    \begin{itemize}
        % $CHARACTHERISTIC{Reduced Facial Muscle Activity; 1}
        \item \textbf{Reduced Facial Muscle Activity}: Schizophrenia is caractherized in some literature with reduced facial muscle activity, which in turn might be responsible for other common charactheristics such as \emph{slowed speech}, \emph{reduced entonation} and \emph{emotionless speech}.
        % $CHARACTHERISTIC{Speech Apathy; 1}
        \item \textbf{Speech Apathy / Emontionless}: Schizophrenia is caractherized with having speech with little to no entonation and variability, which makes it charactherized as emotionless or apathic.
    \end{itemize}

\section{Techniques}
    \begin{itemize}
        % $TECHNIQUE{Spectral Decomposition; 1}
        \item \textbf{Spectral Decomposition (Formants)}: According to the paper, one of the best techniques to analyze speech phonetics is the use of \emph{Spectral Decomposition} which decomposes a sound into \emph{formants} (resonance bands produced while speaking). The resonance \emph{F1} is typically associated with tongue height or openness of mouth and \emph{F2} associated with tongue position from back to front. These values are dependent on the physical structure of the person of interest, so generally, what is analyzed is the variability of the values.
        % $METRIC{DSM Criterion}
        \item \textbf{DSM-IV Criterion}: DSM-IV refers to a manual called \emph{Diagnostic and Statistical Manual of Mental Disorders} on its fourth edition. This manual is one of the best and contributed majorly to improving the reliability of psychiatric diagnosis. \href{https://en.wikipedia.org/wiki/Diagnostic_and_Statistical_Manual_of_Mental_Disorders#DSM-IV_(1994)}{Further Information}
        % $METRIC{Positive and Negative Syndrome Scale}
        \item \textbf{Positive and Negative Syndrome Scale (PANSS)}: A scale created for measuring/classifying the severity of the symptoms of schizophrenia. Recognized as a gold standard for clinicians' evaluation of a schizophrenic patient. \href{https://en.wikipedia.org/wiki/Positive_and_Negative_Syndrome_Scale}{Further Information}
         
    \end{itemize}

\section{Metrics}
    \begin{itemize}
        % $METRIC{F1 Formant}
        \item \textbf{Standard Deviation of F1 Formant}: The standard deviation of the \emph{F1 Formant} was calculated from a one minute sample from the thirty minute recording of each patient. 
        % $METRIC{F2 Formant}
        \item \textbf{Standard Deviation of F2 Formant}: The standard deviation of the \emph{F2 Formant} was calculated from a one minute sample from the thirty minute recording of each patient. 
    \end{itemize}

\section{Problems}
    \begin{itemize}
         % $PROBLEM{Small Sample}
         \item \textbf{Small Sample}: The study has a rather small sample which could have affected greatly the results obtained.
         % $PROBLEM{No Control Group}
         \item \textbf{No Control Group}: The study did not test its hypothesis with a control group, meaning that it did not identify schizophrenic psychosis but instead clinical high risk.
         % $PROBLEM{Difference with other Neurological Disorders}
         \item \textbf{Difference with other Neurological Disorders}: The study did not test its hypothesis with different neurological diseases, which would evaluate if this classifier is effective at identifying the disease or simply the onset of neurological diseases.
         % $PROBLEM{Accoustic Imprecision}
         \item \textbf{Accoustic Imprecision}: The quality of audio recordings was severely impaired, which not only led to the dismissal of some but could also have greatly affected the analysis.
         % $PROBLEM{Innapropriate Interview Type}
         \item \textbf{Semi-Structured Interview Innapropriate}: The phonetic analysis was done from recordings from semi-structured interviews with schizophrenic patients. However, since the study wanted to analyze strictly natural phonetic differences between schizophrenic patients, this is not the vest approach. A more advised technique would be to read out loud a previously stipulated text.
    \end{itemize}


\section{Final Remarks}
    
    Although small both in sample size and extension and complexity of the study, it achieved promising results, showing that there is a negative correlation mainly between \emph{F2} and the score of negative symptoms, with \emph{F1} showing a correlation but under the threshold of significance.

\breakline

\begin{center}
    \section*{Possibly Useful Citations}
\end{center}
\emph{* None found *}

\end{document}
