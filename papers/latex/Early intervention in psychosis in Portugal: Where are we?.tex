\documentclass{Paper_Summary}

% $REF{Early intervention in psychosis in Portugal: Where are we?, Coentre R. Levy P. Acta Medica Portuguesa, (2020), 540-542, 33(9)}
% $TITLE{Early intervention in psychosis in Portugal: Where are we?}
% $AUTHOR{Coentre R. Levy P.}
% $DATE{2020}

% $START-DATE{14/11/2021}
% $END-DATE{14/11/2021}

% ================================== VARIABLES ================================== 
\renewcommand{\varpapertitle}{{Early intervention in psychosis in Portugal: Where are we?}}
\renewcommand{\varpaperauthor}{Coentre R. Levy P.}
\renewcommand{\vardate}{{November 2021}}
% ================================== ========= ================================== 

\begin{document}
\makepapertitle

\breakline

\begin{center}
    \section*{Focus}
\end{center}

    This paper openly discusses the current state of psychosis detection and intervention in Portugal. The authors first compare Portugal to other countries, in terms of team number and sizes, in their efforts to detect psychosis in its early stages.

    The authors state that there has been an increase in effort from the Portuguese government to treat patients who suffer from such disorders in the last two decades. However, steps could still be taken to maintain the same level of treatment as the one from neighboring countries, improve diagnosis and early identification.

    Such steps could include the integration of generalized intervention in a National Mental Health Plan and better use of the multidisciplinary teams' resources. Although not explicitly stated, the better use of teams' resources could involve the use of computerized resources. These computerized resources could support the clinicians in their decisions since, at the moment, they rely on clinical analysis of patients' discourse and self-report of symptoms which could in part be unreliable.

\breakline

\newpage

\section{Psychosis Characteristics}
\emph{* None to discuss, not the objective of this paper *}

\section{Techniques}
\emph{* None to discuss, not the objective of this paper *}

\section{Metrics}
\emph{* None to discuss, not the objective of this paper *}

\section{Problems}
\emph{* None to discuss, not the objective of this paper *}

\section{Final Remarks}
\emph{* None to discuss, not the objective of this paper *}

\breakline

\begin{center}
    \section*{Possibly Useful Citations}
\end{center}
\emph{* None found *}

\end{document}
