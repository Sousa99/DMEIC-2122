\documentclass{Paper_Summary}

% ================================== VARIABLES ================================== 
\renewcommand{\varpapertitle}{{Assessment of Parkinson's disease medication state through automatic speech analysis}}
\renewcommand{\varpaperauthor}{Pompili A. Solera-Ureña R. Abad A et al.}
\renewcommand{\vardate}{{October 2021}}
% ================================== ========= ================================== 

\begin{document}
\makepapertitle

\breakline

\begin{center}
    \section*{Critique}
\end{center}

    This study does not match the focus of the future thesis since it mainly focuses on \emph{Parkinson's Disease} instead of \emph{Psychosis and Schizphrenia}. The main takeaway from this study is that medication states can be identified from speech analysis even if this identification differs from individual to individual.

    The thesis will investigate whether previous findings identified psychosis/schizophrenia or simply the effects of the medication. The knowledge that medication has such different effects from individual to individuals complicates the study since it cannot be directly measured if the features retrieved from psychotic patients match the ones from other neurological diseases.

    Although this knowledge might have arduous repercussions to the study, since the effect is still identifiable with natural language analysis, there must be a possible solution.

    In this study, the analysis was mainly done through \emph{word embeddings} fed to a \emph{neural network}, although both of these techniques will not be used in the thesis, the first because it has already been studied and second because it provides little explanation to the results obtained, it still is interesting to see the results obtained with them.

    The technique that might be interesting for the thesis is the use of one classifier for each subject, therefore creating an implicit baseline for each one.

\breakline

\end{document}
