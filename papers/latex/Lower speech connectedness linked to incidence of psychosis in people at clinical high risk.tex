\documentclass{Paper_Summary}

% $REF{Lower speech connectedness linked to incidence of psychosis in people at clinical high risk, Spencer T. Thompson B. Oliver D. et al. Schizophrenia Research, (2021), 493-501, 228}
% $TITLE{Lower speech connectedness linked to incidence of psychosis in people at clinical high risk}
% $AUTHOR{Spencer T. Thompson B. Oliver D. et al.}
% $DATE{2021}

% $START-DATE{23/10/2021}
% $END-DATE{24/10/2021}

% ================================== VARIABLES ================================== 
\renewcommand{\varpapertitle}{{Lower speech connectedness linked to incidence of psychosis in people at clinical high risk}}
\renewcommand{\varpaperauthor}{Spencer T. Thompson B. Oliver D. et al.}
\renewcommand{\vardate}{{October 2021}}
% ================================== ========= ================================== 

\begin{document}
\makepapertitle

\breakline

\begin{center}
    \section*{Focus}
\end{center}

    This paper focuses on the prediction of future psychosis through the analysis of word and sentence structure. This paper comes as a continuation of the study of \textbf{formal thought disorder} and its identification through biomarkers, specifically coherence biomarkers, which allow for the measurement of \textbf{disorganized speech}.

    Whereas previous studies reported great results using techniques such as \emph{Latent Semantic Analysis}, this study focuses on an alternative technique in order to grasp the connectedness of the subject's speech, \textbf{Word Graphs}.

    The authors had three focus groups during their study:
    \begin{enumerate}
        \item \textbf{Control Group}: Subjects who had not displayed any psychotic event, head trauma and did not have a familiar history that involved neurological diseases.
        \item \textbf{First Episode Psychosis}: Subjects who had already displayed at least one psychotic episode in their life.
        \item \textbf{Clinical High Risk for Psychosis}: Subjects who had not displayed any psychotic event but flagged for possible \emph{clinical-high risk}.
    \end{enumerate}

\breakline

\newpage

\section{Psychosis Characteristics}
    \begin{itemize}
        % $CHARACTHERISTIC{Disruption on Flow of Ideas; 1}
        % $CHARACTHERISTIC{Disruption of Syntax Rules; 1}
        % $CHARACTHERISTIC{Absence of Topic; 1}
        \item \textbf{Formal Thought Disorder}: Initially \emph{formal thought disorder} was only associated with schizophrenia, but in the coming years, it was discovered that it also occurs with affective psychoses, non-psychotic illnesses, and even possibly in healthy controls. Characterized by disruption of the flow of ideas, absence of topic, and loss of coherence.
    \end{itemize}

\section{Techniques}
    \begin{itemize}
        % $TECHNIQUE{Word Graph Analysis; 1}
        \item \textbf{Word Graph Analysis}: Allows for the assessment of topological structures of discourse. These topological structures can then be compared to others to identify possible deviations and reduced discourse coherence. The main advantage of this technique against the more usual and explored \emph{LSA} is that it does not require a large corpus and is quite efficient. Described as the production of directed graphs where words are mapped to nodes and temporal connectedness of words as links between said nodes.
        % $TECHNIQUE{Semi-Structured Interview; 1}
        \item \textbf{Semi-Structured Interview}: The authors displayed eight pictures to each one of the subjects. For each picture, the authors asked the patients to talk for one minute, and if they stopped before, the authors asked to keep talking until the end of the time.
        % $TECHNIQUE{Matching of Control with Schizophrenic; 1}
        \item \textbf{Matching of Control with Schizophrenic Patients}: Authors matched each member of each group to an element of every other group. Matching was done according to their \emph{age}, \emph{gender}.
        % $TECHNIQUE{Structured Interview for Prodromal Syndromes / Scale of Prodromal Symptoms; 0.5}
        \item \textbf{Structured Interview for Prodromal Syndromes/Scale of Prodromal Symptoms (SIPS / SOPS)}: Used in order to evaluate if the patients met the criteria in order to enter the study.
        % $TECHNIQUE{Comprehensive Assessment of At-Risk Mental States; 0.5}
        \item \textbf{Comprehensive Assessment of At-Risk Mental States (CAARMS)}: Assessment generally done through a \emph{semi structured interview} which tries to identify help-seeking clinical high and ultra-high risk youths. \href{https://www.orygen.org.au/Training/Resources/Psychosis/Manuals/CAARMS}{Further Information}
        
    \end{itemize}

\section{Metrics}
    \begin{itemize}
        % $METRIC{Word Graph: Number of Edges}
        \item \textbf{Number of Edges in Word Graph}: The number of edges in a word graph is seen as a good metric to measure speech connectedness.
        % $METRIC{Word Graph: Number of Nodes in LCC}
        \item \textbf{Number Nodes in Largest Connected Component in Word Graph}: The number nodes in the largest \emph{Connected Component} in a word graph is seen as a good metric to measure speech connectedness. \emph{Strong Connected Components} is a group of nodes that are accessible to each other through the directed graph.
        % $METRIC{Word Graph: Number of nodes in LSC}
        \item \textbf{Number Nodes in Largest Strong Connected Component in Word Graph}: The number of nodes in the largest \emph{Strong Connected Component} in a word graph is seen as a good metric to measure speech connectedness. \emph{Strong Connected Components} is a group of nodes that are accessible to each other through the directed graph.
        % $METRIC{Word Graph: Probability of LCC}
        \item \textbf{Probability of the Largest Connected Component}: The probability of the largest \emph{Connected Component} in a word graph occurring, calculated by doing random shuffles of the word in the utterance.
        % $METRIC{Word Graph: Probability of LSC}
        \item \textbf{Probability of the Largest Strongly Connected Component}: The probability of the largest \emph{Strong Connected Component} in a word graph occurring, calculated by doing random shuffles of the word in the utterance.
        % $METRIC{Thought and Language Index}
        \item \textbf{Thought and Language Index (TLI)}: Through the previously described \emph{semi-structured interviews} each transcript is then analyzed for the presence of eight abnormalities. This technique allows for the assessment and scoring of formal thought disorders in patients. The eight abnormalities are:
        \begin{itemize}
            \item Poverty of Speech
            \item Weakening of Goal
            \item Looseness
            \item Peculiar word use
            \item Peculiar sentence construction
            \item Peculiar logic
            \item Perseveration
            \item Distractibility
        \end{itemize}
        \href{https://www.cambridge.org/core/journals/the-british-journal-of-psychiatry/article/thought-and-language-index-an-instrument-for-assessing-thought-and-language-in-schizophrenia/FEFB9ADC871CDFE528B4B87F99A4F054}{Further Information}
        % $METRIC{DSM Criterion}
        \item \textbf{DSM-IV Criterion}: DSM-IV refers to a manual called \emph{Diagnostic and Statistical Manual of Mental Disorders} on its fourth edition. This manual is one of the best and contributed majorly to improving the reliability of psychiatric diagnosis. \href{https://en.wikipedia.org/wiki/Diagnostic_and_Statistical_Manual_of_Mental_Disorders#DSM-IV_(1994)}{Further Information}
         
    \end{itemize}

\section{Problems}
    \begin{itemize}
        % $PROBLEM{Small Sample}
        \item \textbf{Small Sample}: The study has a rather small sample which could have affected greatly the results obtained.
        % $PROBLEM{Patients under Medication}
        \item \textbf{Patients under medication}: Study was carried out with subjects who were taking antipsychotic medication that could have affected the results even if the authors did not find any correlation between the medication dosage and the results.
    \end{itemize}

\section{Final Remarks}

    The authors gathered mainly two conclusions from the study:
    \begin{enumerate}
        \item Speech Graph measures correlate with \emph{TLI} scores, which showed a correct association between both. With computational developments, speech graphs may be a possible diagnostic tool or a support tool for diagnosticians.
        \item Speech Graph measures predicted accurately, the progression from \emph{clinical-high risk} patients into psychoses patients. 
    \end{enumerate}

\breakline

\begin{center}
    \section*{Possibly Useful Citations}
\end{center}
\emph{* None found *}

\end{document}
