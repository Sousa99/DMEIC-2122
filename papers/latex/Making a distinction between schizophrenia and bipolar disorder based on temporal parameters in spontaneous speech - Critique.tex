\documentclass{Paper_Summary}

% ================================== VARIABLES ================================== 
\renewcommand{\varpapertitle}{{Making a distinction between schizophrenia and bipolar disorder based on temporal parameters in spontaneous speech}}
\renewcommand{\varpaperauthor}{Gosztolya G. Bagi A. Szalóki S et al.}
\renewcommand{\vardate}{{October 2021}}
% ================================== ========= ================================== 

\begin{document}
\makepapertitle

\breakline

\begin{center}
    \section*{Critique}
\end{center}

    The techniques used in this paper serve little use to the future thesis work since it focuses mainly on time and word embeddings of the discourse of patients. Although the techniques do not match the ones from the thesis, their results justify that neurological diseases are differentiable among themselves, at least some of them and to some level.

    In the thesis, the stud will record patients who suffer from different disorders but that are medicated with the same or similar anti-psychotics.
    With this expansion, the thesis will evaluate if what was recognized by its previous implementation is the effect of medication or indeed psychosis itself. However, this is based on the premise that disorders are distinguishable between themselves. Of course, that even then, the effects of possible medications might change from person to person or neurological disease. At the very least, this is another factor that would be taken into consideration in the study.

    Another aspect that could be helpful to follow for the thesis is the representation of results in a dimensional space of greater representability for a facilitated interpretation of the results. 

\breakline

\end{document}
