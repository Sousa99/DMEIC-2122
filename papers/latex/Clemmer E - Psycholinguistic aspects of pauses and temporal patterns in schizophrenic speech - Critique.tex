\documentclass{Paper_Summary}

% ================================== VARIABLES ================================== 
\renewcommand{\varpapertitle}{{Psycholinguistic aspects of pauses and temporal patterns in schizophrenic speech}}
\renewcommand{\varpaperauthor}{Clemmer E.}
\renewcommand{\vardate}{{September 2021}}
% ================================== ========= ================================== 

\begin{document}
\makepapertitle

\breakline

\begin{center}
    \section*{Critique}
\end{center}
    
    Although overall the paper is great at discussing the speech metrics that can be used in order to evaluate any pecularities in terms of speech and that eventough the paper states that if the disturbances are cognitive instead of simply spoken I think it would have been interesting to evaluate possibly more direct approaches that could evaluate cognitive pecularities.

    It is important to take into consideration the date at which the paper was release, the techniques and processes in terms of language processment are not at the same level as we are of today, but still, for the experiemnts every recording was transcribed and analysed in detail in terms of syllables, pauses, silences, etc. The pauses were even analysed in terms of where they happened syntatically, so it would have been interesting to get some information and exploration of the recordings in terms of syntatic analysis.
    
    There are also some problems regarding the study, for example small and restrictive sample and possible other major factors which are not propperly addressed in the study. There is the possibility that factors such as medication and lobotomies would affect results.

    It would be interesting to study wether medication is a factor in these types of studies or if could be ignored and the same results could be retrieved from unmedicated patients.

\end{document}
