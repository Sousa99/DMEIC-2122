\documentclass{Paper_Summary}

% $REF{Thought disorder or speech disorder in schizophrenia?, Chaika E, Schizophrenia Bulletin, (1982), 587-591, 8(4)}
% $TITLE{Thought disorder or speech disorder in schizophrenia?}
% $AUTHOR{Chaika E.}
% $DATE{1982}

% $START-DATE{27/09/2021}
% $END-DATE{27/09/2021}

% ================================== VARIABLES ================================== 
\renewcommand{\varpapertitle}{{Thought disorder or speech disorder in schizophrenia?}}
\renewcommand{\varpaperauthor}{Chaika E.}
\renewcommand{\vardate}{{September 2021}}
% ================================== ========= ================================== 

\begin{document}
\makepapertitle

\breakline

\begin{center}
    \section*{Focus}
\end{center}
    This paper does not focus on studying or trying to identify psychosis from a control group, instead it serves as an open discussion as to what is the relation between disordered speech and disordered thought and if any correlation can even be made between these two concepts.

    Overall this paper arguments that if, annotating that there is no evidence (until then) that suggested this, there is a connection between these terms it would certainly not be a direct relation or of 1 to 1.

    The main arguments enumerated by the author that there is not a direct linking between them were:
    \begin{enumerate}
        \item Speech and though are not identical, since one can think of something and its discourse express a completelly different thing or even nothing.
        \item There is speech that does not convey thought at all (\emph{phatic communication} defined by Malinowski), for example greetings.
        \item Even different languages employ different grammatical rules and inherently different structures, meaning that language ahd its corresponding rules aare simply a construct of a society and therefore cannot describe such a universal and humanly concept such as thought.
    \end{enumerate}
    The author enumerates more reasons why thought and speech must be differentiated, but these are more essential and expose the overall logic behind its reasoning.

    The author defends that a good portion of the usual charactheristics for which schizophrenia is identified imply only a speech disorder, and that this might be a valid criterion for the identification of schizophrenia, what does not in fact prove is that schizophrenia should be grouped under thought disorders.
    
    The exception to the rule, made above, are tests in which the meaning, semantics or discourse is evalutated. In these types of tests thought can be somewhat studied, altought taking into consideration that even then is affected by speech processes.

    Ultimately, the author relies on the principle of Occam's razor, stating that the simplest explanation for the fact that most schizophrenic charactheristics are described as speech idiosyncrasies would be that schizophrenia is a speech disorder and should be evaluated, studied and categorized as such.

\breakline

\newpage

\section{Psychosis Characteristics}
    \begin{itemize}
        % $CHARACTHERISTIC{Neologization; 0.5}
        \item \textbf{Neologizing}: to make or use new words or create new meanings for existing words (by definition).
        % $CHARACTHERISTIC{Gibberish; 0.5}
        \item \textbf{Gibberish}: unintelligible or meaningless speech or writing; nonsense (by definition).
        % $CHARACTHERISTIC{Word Salad; 0.5}
        \item \textbf{Word Salad}: a jumble of extremely incoherent speech as sometimes observed in schizophrenia (by defintion).
        % $CHARACTHERISTIC{Abnormal Rhyming; 0.5}
        \item \textbf{Innapropriate Rhymming}: Unexpected rhymming while in apparently normal discourse.
        % $CHARACTHERISTIC{Aliteration; 0.5}
        \item \textbf{Aliterating}: the occurrence of the same letter or sound at the beginning of adjacent or closely connected words (by definition).
    \end{itemize}

\section{Techniques}
    \emph{* None to discuss, not the objective of this paper *}

\section{Metrics}
    \emph{* None to discuss, not the objective of this paper *}

\section{Problems}
    \begin{itemize}
        % $PROBLEM{Difference with other Neurological Disorders}
        \item \textbf{Similarities with Manic patients}: The author states the symptoms / charactheristics of schizophrenia in terms of hallucinations a gibberish sentences tends to approximate those with diagnosis of manic. Also, the author sugests that it would be interesting to study the difference between both.
    \end{itemize}


\section{Final Remarks}
    \emph{* None to discuss, not the objective of this paper *}

\breakline

\begin{center}
    \section*{Possibly Useful Citations}
\end{center}

    \begin{itemize}
        % $CITATION{Thought Disorder VS Speech Disorder}
        \item \textbf{(Chaika E., 1982, p. 589)}: "One's thoughts must be separate from one's speech, preceding it, for how else could one choose the exact words and syntax one whished to convey one's meaning."
    \end{itemize}

\end{document}
