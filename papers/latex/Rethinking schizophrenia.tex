\documentclass{Paper_Summary}

% $REF{Rethinking schizophrenia, Insel T. Nature, (2010), 187-193, 468(7321)}
% $TITLE{Rethinking schizophrenia}
% $AUTHOR{Insel T.}
% $DATE{2010}

% $START-DATE{31/10/2021}
% $END-DATE{01/11/2021}

% ================================== VARIABLES ================================== 
\renewcommand{\varpapertitle}{{Rethinking schizophrenia}}
\renewcommand{\varpaperauthor}{Insel T.}
\renewcommand{\vardate}{{October 2021}}
% ================================== ========= ================================== 

\begin{document}
\makepapertitle

\breakline

\begin{center}
    \section*{Focus}
\end{center}

    This paper, differently than most of the papers discussed, discusses the \textbf{current state} of schizophrenia diagnosis and possible \textbf{possible developments} for the future of its diagnosis, prevention, and improvement of conditions.

    Starting by disclosing how the unimproved the branch of mental illnesses and their understatement is. According to the authors:
    \begin{itemize}
        \item Only \textbf{14\%} of short term recovery, and \textbf{16\%} long-term recovery.
        \item \textbf{Less than 20\%} of the people diagnosed with schizophrenia are employed.
        \item \textbf{More than 20\%} of the people diagnosed with schizophrenia are homeless.
        \item \textbf{Three times more} likely to be incarcerated or with a criminal record.
    \end{itemize} 

    For the reasons stated above, an early diagnosis is crucial, and its current methodology should be analyzed.

    The history of the diagnosis of schizophrenia can be divided into the following periods:
    \begin{enumerate}
        \item Started in the 19th century, by being defined as \textbf{dementia preacox}, directly translated to an antecedent stage of dementia. When the term \emph{schizophrenia} was coined, the diagnosis and improvement focused only on a reteaching/pediatric methodology, and patients were seen as with a problem that could be easily solved and originated simply by an infancy trauma. Although research mentioned the brain, it did not use it as a possible origin for the illness. Studies focused solely on 
        \item The second stage, at the second half of the 20th century, was marked by the complete opposite approach. Illness was seen as partially if not totally curable by medication, by using \emph{neuropletic and antipsychotic drugs} and taking advantage of brain chemistry. Although this proved to be useful when dealing with the denominated \emph{positive symptoms}, it disregarded the \emph{mind} aspect of the brain and therefore most of the \emph{negative symptoms}.
        \item Recent studies have focused on the analysis of the illness as possibly originated from multiple factors. Childhood traumas, maldevelopments of the brain, and heritable factors are all considered possible originating factors for the disease. Of note that such studies are still very recent:
        \begin{itemize}
            \item Traumas are not a reliable factor since some people that undergo them, at later stages in life, develop schizophrenia, and others do not.
            \item Maldevelopmentds of the brian such as on the \emph{NMDA} receptors seem to correlate but are not present in all subjects.
            \item Some studies support the idea of heritability. However, even in monozygotic twins, concordance with the theory is around 50\% and not the expected 100\% for a Mendelian disorder.
        \end{itemize}
    \end{enumerate}
    The author briefly touches on the possibility that what is considered schizophrenia nowadays is merely a merging of various illnesses, which would somewhat justify the heterogeneity among schizophrenic symptoms and originating factors. 

    The authors noted that one problem when studying mental illnesses is that they cannot be modeled into animals in a relatively quick and easy way. Studies have tried to alter the neuro mapping of animals at an early stage of life and evaluated the neuro deficiency progress throughout time. However, the state, behavioral deficiencies, and progression of such a disease cannot be evaluated.

    Four different stages can be identified in the development of schizophrenia:
    \begin{enumerate}
        \item \textbf{Pre-Symptomatic Risk}: Marked by undetectable symptoms or abnormalities, although some outlines have already been outlined based on behavioral and cognitive problems during childhood. Diagnosis can only be made, if at all, through genetic sequencing or analysis of family history.
        \item \textbf{Psychotic Prodrome}: Established by \emph{McGorry}, sometimes also defined as \emph{high or ultra-high risk}, can be identified through \emph{changes of though}, \emph{social isolation} and \emph{functional impairments}, which can sometimes be misintreperted as adolescence patterns. For this differentiation the \emph{Structured Interview for Prodromal Syndromes (SIPS)} can be employed. The diagnosis can be aided through the use of other metrics as \emph{neuroimaging}, \emph{biomarkers} and \emph{time analysis}.
        \item \textbf{Acute Psychosis}: Defined mostly by the display of \emph{positive symptoms} with some, although less pronounced, \emph{negative symptoms} such loss of mobility and cognitive deficiency.
        \item \textbf{Chronic Illness}: The problem transforms from a solely psychiatric illness into a full-fledged medical illness. The symptoms previously discussed are more prevalent and worsened, having already affected the patient's integration within society, usually leading to unhealthy habits (such as smoking or eating in excess). The patient may even require reintegration services in order to be maintained.
    \end{enumerate}

    Finally, the authors conclude by enumerating in which domains schizophrenia diagnosis and treatment can improve in the future:
    \begin{enumerate}
        \item \textbf{Schizophrenia prevention}: The first effort must come from a prevention point of view. Although a one-cure-all medication would be an incredible breakthrough, its existence is not guaranteed, and efforts must be made from the perspective of understanding early signs of the illness and possible reversion of the neural abnormalities.
        \item \textbf{Reducing the cognitive deficits}: Although medication for management of so-called \textbf{positive symptoms}, characterized by hallucinations and delusions, medication focused on reducing \textbf{negative symptoms} is severely underinvestigated. Such medication would aid the social inclusion of these members in society.
        \item \textbf{Care integration}: At the moment, medical care and psychiatric care are two distinct and isolated domains, which aggravates the state of schizophrenia prevention. The authors wish that mental illnesses' care was integrated akin to the prevention and care for diabetes or other chronic diseases.
        \item \textbf{Schizophrenia stigma}: Progress has been made from the time that diagnosed patients were placed in institutions similar to items sent to a warehouse. Still, there is space for progress when it comes to \textbf{stigma} and the presumption of fatality of mental illnesses.
    \end{enumerate} 

\breakline

\newpage

\section{Psychosis Characteristics}
\emph{* None to discuss, not the objective of this paper, but mentioned above *}

\section{Techniques}
\emph{* None to discuss, not the objective of this paper *}

\section{Metrics}
\emph{* None to discuss, not the objective of this paper *}

\section{Problems}
\emph{* None to discuss, not the objective of this paper *}

\section{Final Remarks}
\emph{* None to discuss, not the objective of this paper *}

\breakline

\begin{center}
    \section*{Possibly Useful Citations}
\end{center}

    \begin{itemize}
        % $CITATION{Rethinking Schizophrenia}
        \item \textbf{(Insel T., 2010, p. 191)}: "refocusing our approach to schizophrenia on early detection and early intervention could yield substantial improvements in outcomes over the next decade or two. This will, of course, require sensitive and specific diagnostic tools as well as safe and effective interventions."
    \end{itemize}

\end{document}
