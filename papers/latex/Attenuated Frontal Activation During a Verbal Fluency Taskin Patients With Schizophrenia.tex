\documentclass{Paper_Summary}

% $REF{Attenuated Frontal Activation During a Verbal Fluency Task in Patients With Schizophrenia, Curtis V, Bullmore E, Brammer M, et al. (1998)}
% $TITLE{Attenuated Frontal Activation During a Verbal Fluency Taskin Patients With Schizophrenia}
% $AUTHOR{Curtis V, Bullmore E, Brammer M, et al.}
% $DATE{1998}

% $START-DATE{03/10/2021}
% $END-DATE{04/10/2021}

% ================================== VARIABLES ================================== 
\renewcommand{\varpapertitle}{{Attenuated Frontal Activation During a Verbal Fluency Taskin Patients With Schizophrenia}}
\renewcommand{\varpaperauthor}{Curtis V, Bullmore E, Brammer M, et al.}
\renewcommand{\vardate}{{October 2021}}
% ================================== ========= ================================== 

\begin{document}
\makepapertitle

\breakline

\begin{center}
    \section*{Focus}
\end{center}
    
    This paper revolves around \emph{Magnetic Resonance Imaging (MRI)} analysis on schizophrenic patients, trying to recreate the methodology from previous papers that used \emph{Positron Emission Tomography (PET)} and showed satisfying and consistent results.

    The paper starts by enumerating the multiple advantages of the employment of MRI's instead of PET scans:
    \begin{enumerate}
        \item There are no long-term repercussions from the test. PET scans rely on radioactivity, MRIs rely on the measurement of blood-oxygen levels at the level of neurons, so there is a clear choice in terms of health.
        \item MRIs have a greater temporal and spatial resolution of imaging than PET scans. MRI's resolution allows for the identification and measurement of unique voxels in the human brain.
    \end{enumerate}
    While there is a clear rationale behind the choice of MRIs, there is one main limitation: extremely sensitive to perturbation during the process of imaging, going from slight involuntary movements of the patient to the extreme of being perturbated by biological movements (such as breathing or pulsing).

    The methodology revolves around the main assessment phase. During this main phase, the volunteer will try to enumerate as many items (possibly of a category) as possible, but there are multiple possible formats for this assessment:
    \begin{itemize}
        \item \emph{Auditory / Visual}: the cues, either for displaying the category of items or when the volunteer should start the test, can be given auditorial or visually. - The paper chose \emph{Visual}.
        \item \emph{Paced / Unpaced}: If the volunteers should state an item at a given time interval, for example, every three seconds state one item, or if instead there is a larger interval of time during which the volunteers can enumerate as many items as possible. - The paper chose the first, \emph{Paced}.
        \item \emph{Overt / Covert}: If the volunteers should enumerate or repeat the word verbally out loud or to themselves. - The paper chose the second, \emph{Covert}.
        \item \emph{Random / Constrained}: If the volunteers can enumerate anything that comes to mind or if the items enumerated should all belong to a specific category or respect a specific criterion. - The paper chose the second, \emph{Constrained}, in this case, words starting with a given letter.
    \end{itemize}

    As a control, besides having volunteers that are not diagnosed with schizophrenia, the study asks that the test subjects between assessments rest. During this phase, they must repeat, every 3 seconds, the word \emph{rest}, during which the MRI machine is still running. 

\breakline

\newpage

\section{Psychosis Characteristics}
    \begin{itemize}
        % $CHARACTHERISTIC{Cerebral Frontal Region Attenuated; 1}
        \item \textbf{Cerebral Frontal Region Attenuated CBF}: When comapared to controls, schizophrenic patients displayed attenauted cerebral frontal region CBF, during the word generation phase.
        % $CHARACTHERISTIC{Medial Parietal Cortex Strengthen; 1}
        \item \textbf{Cerebral Medial Parietal Cortex Strengthened}: When comapared to controls, schizophrenic patients displayed strengthened cerebral medial pariental cortex region CBF, during the word repetition / rest phase.
    \end{itemize}

\section{Techniques}
    \begin{itemize}
        % $TECHNIQUE{Magnetic Resonance Imaging; 1}
        \item \textbf{Magnetic Resonance Imaging (MRI)}: The study uses MRI in order to analyze the variation in oxygenated blood flow on the brain, in its various areas, between schizophrenic patients and controls. Although it does not allow directly to measure the variation in the amount of flow between two images, only the location and timing, through a process called \emph{ANOVA} this can be estimated.
        % $TECHNIQUE{Positron Emission Tomography; 0}
        \item \textbf{Positron Emission Tomography (PET)}: Although this study does not employ this technology, it follows the methodology used by other papers that in turn used PET scans, so it is worth mentioning.
        % $TECHNIQUE{Matching of Control with Schizophrenic; 1}
        \item \textbf{Matching of Control with Schizophrenic Patients}: Each schizophrenic patient was matched with a corresponding control. Matching was done according to their \emph{age}, \emph{handiness (right or left-handed)}, \emph{IQ}, \emph{ability to perform the verbal fluency test}.
        % $TECHNIQUE{Verbal Fluency Test; 0.5}
        \item \textbf{Verbal Fluency Test}: The methodology is based on the FAS Verbal Fluency Test, where the subjects are asked to enumerate a given set of words constrained to a certain condition.
        
    \end{itemize}

\section{Metrics}
    \begin{itemize}
        % $METRIC{Cerabral Blood Flow}
        \item \textbf{Cerebral Blood Flow (CBF)}: The amount of oxygenated blood flow that can be measured through MRI imaging.
        % $METRIC{Scale  for  Assessment  of  Thought,  Language, and Communication}
        \item \textbf{Scale for Assessment of Thought, Language, and Communication}: List of twenty subscales which are then evaluated in a 5 points scale (with the exception of items 10-18 which are evaluated in a 4 point scale). The definition of a scale provides a brief description of the subscale and characterizes what each value in that subscale equates to. \href{https://www.northeastern.edu/cali/wp-content/uploads/2017/03/Scale-for-the-assessment-of-thought-language-and-communication.pdf}{Further Information}
        % $METRIC{Schedule for Affective Disorders and Schizophrenia}
        \item \textbf{Schedule for Affective Disorders and Schizophrenia}: Collection of Psychiatric diagnostic criteria composed of rating scales. Each scale can be scored from zero to three. — \href{https://en.wikipedia.org/wiki/Schedule_for_Affective_Disorders_and_Schizophrenia}{Further Information}
        % $METRIC{Scale for the Assessment of Positive Symptoms}
        \item \textbf{Scale for the Assessment of Positive Symptoms}: Scale used to measure positive symptoms of schizophrenia. Each domain/subscale is measure from zero to five. — \href{https://en.wikipedia.org/wiki/Scale_for_the_Assessment_of_Positive_Symptoms}{Further Information}
        % $METRIC{Scale for the Assessment of Negative Symptoms}
        \item \textbf{Scale for the Assessment of Negative Symptoms}: Scale used to measure negative symptoms of schizophrenia. Each domain/subscale is measure from zero to five. — \href{https://en.wikipedia.org/wiki/Scale_for_the_Assessment_of_Negative_Symptoms}{Further Information}
        % $METRIC{DSM Criterion}
        \item \textbf{DSM-IV Criterion}: DSM-IV refers to a manual called \emph{Diagnostic and Statistical Manual of Mental Disorders} on its fourth edition. This manual is one of the best and contributed majorly to improving the reliability of psychiatric diagnosis. \href{https://en.wikipedia.org/wiki/Diagnostic_and_Statistical_Manual_of_Mental_Disorders#DSM-IV_(1994)}{Further Information}
         
    \end{itemize}

\section{Problems}
    \begin{itemize}
        % $PROBLEM{Small Sample}
        \item \textbf{Small Sample}: Although understandable due to the high complexity of the study, the sample is constituted by five controls and five schizophrenic patients.
        % $PROBLEM{MRI Account for Perturbances}
        \item \textbf{MRI Account for perturbances}: MRIs are extremely sensitive, and although the test subjects are executing the task covertly, nothing guarantees or accounts for the possible involuntary movements.
        % $PROBLEM{Patients under Medication}
        \item \textbf{Patients under medication}: Study was carried out exclusively with patients taking medication, even though the paper states that there are no registered side effects that would have influenced the results.
        
    \end{itemize}


\section{Final Remarks}
    
    The study was unable of prooving that there is an alteration of the frontotemporal cerebral blood flow or that this fact has any direct influence on the patient's discourse.
    However, the test showed convincing evidence that suggests some attenuated effects on the frontal region and some strengthened medial parietal cortex.

\breakline

\begin{center}
    \section*{Possibly Useful Citations}
\end{center}
    \emph{* None found *}

\end{document}
