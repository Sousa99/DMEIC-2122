\documentclass{Paper_Summary}

% ================================== VARIABLES ================================== 
\renewcommand{\varpapertitle}{{Speech Fluency and Schizophrenic Negative Signs}}
\renewcommand{\varpaperauthor}{Alpert M, Kotsaftis A, Pouget E.}
\renewcommand{\vardate}{{October 2021}}
% ================================== ========= ================================== 

\begin{document}
\makepapertitle

\breakline

\begin{center}
    \section*{Critique}
\end{center}
    
    The paper touches on possible identification of charactheristics and measurement of schizophrenia and its negative symptoms. Although it ends up focusing more on the clinicians' capability of implicitly and almost unknowningly identifying these charactheristics.

    Although this approach aligns somewhat with the approach that I will be following, that the main objetive is not to substitute clinicians diagnosis but instead support them in their decisions, trying to identify possibly charactheristics that are unnoticeable.

    However, what does not coincide between this study and the approach that I will be following it the extreme focus on time measurements instead of pragamatics and discourse analysis. It is not out of the question to use this metrics as support for the study, but the most important analysis would be the one done through context.

    Besides what I identiifed as possible downfalls of the paper, it relies too much on a little subset of concrete values and techniques. It uses only two clinical measurement scales, pause measurement and latency measurement. More metrics and possible analysis of the interdependencies between them.

\breakline

\end{document}
