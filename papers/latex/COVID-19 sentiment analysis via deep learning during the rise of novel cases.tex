\documentclass{Paper_Summary}

% $REF{COVID-19 sentiment analysis via deep learning during the rise of novel cases, Chandra R. Krishna A. PLoS ONE, (2021), 16(8 August)}
% $TITLE{COVID-19 sentiment analysis via deep learning during the rise of novel cases}
% $AUTHOR{Chandra R. Krishna A.}
% $DATE{2021}

% $START-DATE{28/10/2021}
% $END-DATE{29/10/2021}

% ================================== VARIABLES ================================== 
\renewcommand{\varpapertitle}{{COVID-19 sentiment analysis via deep learning during the rise of novel cases}}
\renewcommand{\varpaperauthor}{Chandra R. Krishna A.}
\renewcommand{\vardate}{{October 2021}}
% ================================== ========= ================================== 

\begin{document}
\makepapertitle

\breakline

\begin{center}
    \section*{Focus}
\end{center}

    This paper focused on the study of the progression of \textbf{sentiment} during the peak of cases of \emph{COVID-19} cases in India in a few selected areas. The study analyzed the sentiment expressed by the population through publicly accessible tweets from the general population.

    The study analyzed sentiment from expressed pieces of information, in this case, tweets which were then converted into \textbf{word embeddings} through the use of an already developed tool \textbf{GLoVe}. The authors then fed this \emph{word embeddings} to a \emph{multi-label classifier model} which guessed as to which (possibly choosing more than one) emotion was being expressed with the piece of information.

    The studied possible emotions were:\emph{optimistic}, \emph{thankful}, \emph{empathetic}, \emph{pessimistic}, \emph{anxious}, \emph{sad}, \emph{annoyed}, \emph{denial}, \emph{official report}, \emph{surprise}, \emph{joking}.

    The three models tested on the study:
    \begin{itemize}
        \item \textbf{Long Short-Term Memory RNN (LSTM)}: In theory \textbf{Recurrent Neural Networks} are capable of dealing better with \emph{Natural Language Processing} since they classify each inputted element on the basis of this element and \emph{theoretically} on every element that preceded this one. In practice, RNNs are only capable of correctly representing a few preceding elements. \textbf{LSTM} solves this problem. The authors do not discuss the architecture of LSTMs, but they grasp the preceding words in a superior manner. 
        \item \textbf{Bidirectional Long Short-Term Memory RNN (BD-LSTM)}: There is yet another problem that affects the results obtained when studying \emph{Natural Language Processing}. In Natural Language, words can be affected by \textbf{preceding} but also by \textbf{proceding} words. \textbf{BD-LSTM} architectures take advantage of these two effects on any given element: the effect by preceding and proceeding elements.
        \item \textbf{Bi-Directional Encoder Representations from Transformers (BERT)}: The authors do not describe in detail the architecture of how it works, so further research is necessary to understand the inner workings. But it is a \emph{state-of-the-art} architecture that takes into account the \emph{bidirectional} property of \emph{Natural Language Processing}.
    \end{itemize}

    Both \textbf{LSTM} and \textbf{BD-LSTM} were composed of:
    \begin{itemize}
        \item A \textbf{300 nodes} input layer.
        \item Two hidden layer with \textbf{128 nodes} and \textbf{64 nodes}.
        \item A output layer with \textbf{11 nodes}, each corresponding to one of the eleven sentiments.
    \end{itemize}

    Important to note that the tweets used in the study, before being converted into a \emph{word embedding} its first \textbf{pre-processed}. One of the main steps of the \emph{pre-processment} was the transformation of emoticon icons (emojis) into corresponding words or words.

\breakline

\newpage

\section{Psychosis Characteristics}
\emph{* None to discuss, not the objective of the paper *}

\section{Techniques}
\emph{* None to discuss, not the objective of the paper *}

\section{Metrics}
\emph{* None to discuss, not the objective of the paper *}

\section{Problems}
\emph{* None to discuss, not the objective of the paper *}


\section{Final Remarks}
\emph{* None to discuss, not the objective of the paper *}

\breakline

\begin{center}
    \section*{Possibly Useful Citations}
\end{center}
\emph{* None found *}

\end{document}
