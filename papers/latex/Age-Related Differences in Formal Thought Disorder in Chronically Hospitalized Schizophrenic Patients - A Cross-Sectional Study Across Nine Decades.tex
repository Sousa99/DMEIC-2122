\documentclass{Paper_Summary}

% $REF{Age-Related Differences in Formal Thought Disorder in Chronically Hospitalized Schizophrenic Patients: A Cross-Sectional Study Across Nine Decades, Harvey F, Leibman L, Al E. (1997), 205}
% $TITLE{Age-Related Differences in Formal Thought Disorderin Chronically Hospitalized Schizophrenic Patients:A Cross-Sectional Study Across Nine Decades}
% $AUTHOR{Harvey F, Leibman L, Al E.}
% $DATE{1997}

% $START-DATE{29/09/2021}
% $END-DATE{30/09/2021}

% ================================== VARIABLES ================================== 
\renewcommand{\varpapertitle}{{Age-Related Differences in Formal Thought Disorderin Chronically Hospitalized Schizophrenic Patients:A Cross-Sectional Study Across Nine Decades}}
\renewcommand{\varpaperauthor}{Harvey F, Leibman L, Al E.}
\renewcommand{\vardate}{{September 2021}}
% ================================== ========= ================================== 

\begin{document}
\makepapertitle

\breakline

\begin{center}
    \section*{Focus}
\end{center}
    This paper has its main focus to understand the progression of the disorder of formal thought in schizophrenic patients. The study then according to its methodology later discussed, evaluated the main differences between younger patients, still in the phase of early onset of schizophrenia against geriatric patients.

    The study ended up focusing on mainly two characteristics of thought disorder:
    \begin{enumerate}
        \item Poverty of Speech: less spoken words and less complexity of words spoken
        \item Disconected Speech: described as disconnected thoughts and discourse without a guiding thread.
    \end{enumerate}

    Although the study in itself is not highly complex, it has a pretty substancial sample of patients and discusses various scales and tests done to the patients in order to evaluate their thought disorder and the calculations in order to achieve its conclusions.

\breakline

\newpage

\section{Psychosis Characteristics}
    \begin{itemize}
        % $CHARACTHERISTIC{Poverty of Speech; 1}
        \item \textbf{Deficient verbal productivity}: As previously mentioned, schizophrenic patients even in remission display poverty of speech, talking less and having reduced complexity of spoken words.
        % $CHARACTHERISTIC{Disruption on Flow of Ideas; 1}
        \item \textbf{Disconected Speech}: Speech appears to have no correlation amongst its components. Discourse loses sense and meaning, and difficult to intrpert.
        % $CHARACTHERISTIC{Distractibility; 0.5}
        \item \textbf{Distractibility}: Associated with the disconnected component of formal thought.
        % $CHARACTHERISTIC{Absence of Self Monitoring; 0.5}
        \item \textbf{Reality Monitoring Deficit}: Also associated with the disconnected component of formal thought.
    \end{itemize}

\section{Techniques}
    \begin{itemize}
        % $TECHNIQUE{Diagnostic Interview; 1}
        \item \textbf{Diagnostic Interview}: Patients were interviewed with the specific purpose of evaluating / scoring them against well known scales (discussed in metrics).
        % $TECHNIQUE{Clinical History Chart Review; 1}
        \item \textbf{Clinical History Chart Review}: Patients clinical history pertaining to mental hillnesses was studied and analyzed.
        % $TECHNIQUE{Open Interview; 1}
        \item \textbf{Open Interview}: Patients were individually interviewed in order to get a better grasp of their current situation.
        % $TECHNIQUE{Open Interview with Caregivers; 1}
        \item \textbf{Open Interview with Caregivers}: Patients' caregivers were individually interviewed in order to get a better grasp of their patient's current situation.
    \end{itemize}

\section{Metrics}
    \begin{itemize}
        % $METRIC{Scale  for  Assessment  of  Thought,  Language, and Communication}
        \item \textbf{Scale for Assessment of Thought, Language, and Communication}: List of twenty subscales which are then evaluated in a 5 points scale (with esception of items 10-18 which are evaluated in a 4 point scale). The definition of a scale not only provides a brief description of the subscale but also charactherizes what each value in that subscale equates to. \href{https://www.northeastern.edu/cali/wp-content/uploads/2017/03/Scale-for-the-assessment-of-thought-language-and-communication.pdf}{Further Information} 
        % $METRIC{Mini-Mental  State  examination}
        \item \textbf{Mini-Mental State Examination}: Test to evaluate mental capabilities that was carried out in geryatric patients. Similar to the questions that are poppulariuly asked to people who have just suffered a concussion in order to evaluate their state of confusion and possibly dementia. \href{https://en.wikipedia.org/wiki/Mini%E2%80%93Mental_State_Examination}{Further Information} 
        % $METRIC{DSM Criterion}
        \item \textbf{DSM-III-R Criterion}: DSM-III-R refers to a manual called \emph{Diagnostic and Statistical Manual of Mental Disorders} on its third edition and revised. Its one of the best known publications, and contributted majorily into improving the reliability of psychiatric diagnosis. \href{https://en.wikipedia.org/wiki/Diagnostic_and_Statistical_Manual_of_Mental_Disorders#DSM-III-R_(1987)}{Further Information}
    \end{itemize}

\section{Problems}
    \begin{itemize}
        % $PROBLEM{No Control Group}
        \item \textbf{No Control Group}: Although it is not the main focus of the paper to differentiate formal thought from normals (or a control group) even the paper admits that it would have been helpful.
        % $PROBLEM{Different Environments}
        \item \textbf{Different Environments}: The paper compares two subsets of patients which display formal thought disorders, but since each group was in different institutions the results obtained could have as origin this aspect.
    \end{itemize}


\section{Final Remarks}
    
    The results obtained were the initially hypothesized by the authors:
    \begin{itemize}
        \item Higher severity of negative symptons in old geryatric patients.
        \item Reductions in the severity of positiveand disorganized symptoms.
    \end{itemize}

\end{document}
