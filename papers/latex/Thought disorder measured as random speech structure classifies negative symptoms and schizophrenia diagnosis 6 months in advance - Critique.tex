\documentclass{Paper_Summary}

% ================================== VARIABLES ================================== 
\renewcommand{\varpapertitle}{{Thought disorder measured as random speech structure classifies negative symptoms and schizophrenia diagnosis 6 months in advance}}
\renewcommand{\varpaperauthor}{Mota N. Copelli M. Ribeiro S.}
\renewcommand{\vardate}{{October 2021}}
% ================================== ========= ================================== 

\begin{document}
\makepapertitle

\breakline

\begin{center}
    \section*{Critique}
\end{center}

    The study discussed proves to be a double-edged sword to the future thesis work, mainly for two reasons:
    \begin{enumerate}
        \item It closely follows the approach that could be followed, at least in part, of the thesis work. It analyzes discourse structure (as well as coherence) from audio recordings of patients through the use of \textbf{word graphs} and metrics retrieved from them.
        \item It follows so closely that the thesis' work would lose by re-exploring a theme that has already been explored previously, worsened by the fact that this study was carried out with \emph{Portuguese} speakers even if from Brazil.
    \end{enumerate}

    Nonetheless, the thesis' work could explore this topic of word graphs from various perspectives that would still be useful:
    \begin{enumerate}
        \item First work to study the possibility of \textbf{word graphs} as a diagnostic technique carried out in \emph{Portuguese} speakers of Portugal.
        \item Verify results obtained from previous work, as a follow-up:
        \begin{enumerate}
            \item To the previous thesis' work that served as a precedent to this one, by verifying whether the results obtained for coherence using \emph{LSA} match the ones obtained through this methodology.
            \item To the study described in this paper, by having a larger sample size and focusing on the identification instead of schizophrenic diagnosis prediction.
        \end{enumerate}
        \item Study the possibility of the use of \emph{word graphs} as a diagnostic technique carried out with other techniques for record gathering. The methodology discussed in the paper relies more on personal interaction with the patient by exploring its memories and dreams, whereas the thesis' work would analyze strictly effective image analysis. 
        \item Possibly identifying other metrics or techniques which use \emph{word graphs} as representation.
    \end{enumerate}

    The study had other interesting aspects from this paper that are worth mentioning:
    \begin{enumerate}
        \item The authors developed a classification index which follows more closely clinicians methodology when diagnosing.
        \item The authors tested out \emph{confounding factors} for factors that were annotated but not matched between patients. 
    \end{enumerate}

\breakline

\end{document}
