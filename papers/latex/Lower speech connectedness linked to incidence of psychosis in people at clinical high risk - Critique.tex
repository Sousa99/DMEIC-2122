\documentclass{Paper_Summary}

% ================================== VARIABLES ================================== 
\renewcommand{\varpapertitle}{{Lower speech connectedness linked to incidence of psychosis in people at clinical high risk}}
\renewcommand{\varpaperauthor}{Spencer T. Thompson B. Oliver D. et al.}
\renewcommand{\vardate}{{October 2021}}
% ================================== ========= ================================== 

\begin{document}
\makepapertitle

\breakline

\begin{center}
    \section*{Critique}
\end{center}

    This paper actively discusses the advantages of speech graphs against other techniques and further explains which measures can be extrapolated from speech graphs.

    Certainly, \textbf{speech graphs} has an advantage against its counterpart \textbf{Latent Semantic Analysis} since it does not require a large corpus in order for the study and subsequent measures to be analyzed.

    Coherence has already been analyzed in the previous thesis' work. Even then, it could be advantageous to analyze text coherence through a different technique and evaluate whether the results previously obtained match (even if only in part) the results obtained through this measure.

    The technique used for speech graphs is strikingly efficient, which could provide a live classification of the spoken discourse to help clinicians' diagnosis.

    The thesis could explore this technique in almost all tests (with exception for the enumeration of words with a certain condition and the reading of a story) since it simply epxlores the coherence but the most suited would probably be the retelling of the story of "3 Porquinhos" since it provides longer utterances and discourse.

\breakline

\end{document}
