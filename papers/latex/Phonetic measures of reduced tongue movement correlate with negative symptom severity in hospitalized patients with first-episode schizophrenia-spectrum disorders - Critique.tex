\documentclass{Paper_Summary}

% ================================== VARIABLES ================================== 
\renewcommand{\varpapertitle}{{Phonetic measures of reduced tongue movement correlate with negative symptom severity in hospitalized patients with first-episode schizophrenia-spectrum disorders}}
\renewcommand{\varpaperauthor}{Covington M. Lunden S. Cristofaro S. et al.}
\renewcommand{\vardate}{{October 2021}}
% ================================== ========= ================================== 

\begin{document}
\makepapertitle

\breakline

\begin{center}
    \section*{Critique}
\end{center}

    This study focuses on the phonetic analysis of the recordings, which is a domain that might be interesting to explore during the thesis. This study serves as a great starting point for exploring the phonetic and sound characteristics of the recordings.

    Important to note that many papers for a long time have tried (and been successful) at analyzing characteristics such as pitch and pauses duration. However, the technique \textbf{Spectral Decomposition} although still associated with the mentioned sound characteristics, works in a completely different manner. This technique tries to analyze something nearly or even imperceptible to a clinician's diagnosis requiring heavy computation, identifying metrics that correspond to attributes and characteristics of the patient but that cannot be directly measured.

    Although a reliable technique, it would not allow for mass exploration of the topic and its problems due to its directness and low complexity.
    The low complexity of the technique might have been the reason for the reduced complexity of the study.

    This study has its limitations, but the thesis would surpass some of these limitations:
    \begin{itemize}
        \item \underline{Small Sample}: The thesis already has recordings available for the exploration, and the dimension is better than the one from this study.
        \item \underline{No Control Group}: The thesis has a control group that would allow for the comparison of the results obtained. However, the recording for the thesis has no information regarding the severity of the patients' symptoms.
        \item \underline{Difference with other Neurological Disorders}: The thesis hopes to record more patients who suffer from different neurological disorders, allowing for the differentiation of schizophrenic patients from other groups and analyze the effect of medication.
        \item \underline{Accoustic Imprecision}: The microphone used has enough precision for this type of study.
        \item \underline{Semi-Structured Interview Innapropriate}: During the recordings, the patients have already been asked to read out loud the "O Vento Norte e o Sol" story, which is more appropriate for the technique here used.
    \end{itemize}

\breakline

\end{document}
