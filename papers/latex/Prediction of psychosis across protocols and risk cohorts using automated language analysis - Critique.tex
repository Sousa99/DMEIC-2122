\documentclass{Paper_Summary}

% ================================== VARIABLES ================================== 
\renewcommand{\varpapertitle}{{Prediction of psychosis across protocols and risk cohorts using automated language analysis}}
\renewcommand{\varpaperauthor}{Corcoran C. Carrillo F. Fernández-Slezak D. et al.}
\renewcommand{\vardate}{{October 2021}}
% ================================== ========= ================================== 

\begin{document}
\makepapertitle

\breakline

\begin{center}
    \section*{Critique}
\end{center}

    This paper follows relatively close the structure, methodology, and techniques of the previous \emph{Automated analysis of free speech predicts psychosis onset in high-risk youths}. Therefore comes with almost all the advantages previously mentioned in the critique of the paper.

    The focus of this study remains aligned when it handles semantic context and its coherence. However, it even approximates better what will be the focus of the thesis since it has a brief commentary on the classifier's ability to distinguish schizophrenic patients from normals/controls.

    The paper still references a problem in sample size. The paper does not mention possible medications that the patients were under the influence, and even if there is a difference in medication for \emph{CHR+} and \emph{CHR-} patients.

    Still, there is room for the thesis to expand on: mainly the language (since it will analyze Portuguese) and possible mesh of multiple techniques, analyzing coherence, meaning, and possibly even sound analysis.

\breakline

\end{document}
