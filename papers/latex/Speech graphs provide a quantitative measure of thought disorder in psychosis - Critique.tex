\documentclass{Paper_Summary}

% ================================== VARIABLES ================================== 
\renewcommand{\varpapertitle}{{Speech graphs provide a quantitative measure of thought disorder in psychosis}}
\renewcommand{\varpaperauthor}{Mota N. Vasconcelos N. Lemos N et al.}
\renewcommand{\vardate}{{October 2021}}
% ================================== ========= ================================== 

\begin{document}
\makepapertitle

\breakline

\begin{center}
    \section*{Critique}
\end{center}

    Similar to previous papers by the same authors, the use of \textbf{Word Graphs} could provide an interesting analysis of discourse structure and differentiation between several groups.

    Although this study has a bigger correspondence to the thesis' work since it aims at identifying and differentiating between several groups of which some are diagnosed with psychotic disorder, and another is a control group, still there are some differences:
    \begin{itemize}
        \item While the study tries to differentiate all three groups from the remaining ones, the thesis' work will identify schizophrenic patients from all remaining groups.
        \item The authors did not study possible effects of medication, their types, and their dosage across the various patients. The thesis work will try to examine this aspect.
        \item The thesis will try to expand besides focusing only on discourse structure.
        \item The thesis does not have or intend to have \emph{dream reports} or \emph{clinicians scales and evaluations} into consideration since this would:
        \begin{enumerate}
            \item Break the breach of data protection and the agreed terms of the already done recordings.
            \item Extend into the private and clinicians' domain instead of focusing on more objective and less personal information.
        \end{enumerate}
    \end{itemize}

    Although the strategy employed by the study did not follow their conclusion, the authors state that the discoveries made prove that in the future, clinicians' diagnosis will likely be aided by such technologies as the ones described and not entirely replaced.

    Important to note that this study once again had a relatively small sample, which might have affected results, but the future thesis' work already has a larger sample size.

    This method relies on a more open, long, and complex discourse to evaluate such metrics as the ones discussed. Therefore there is only one option to where it can be employed in the thesis' work, the \textbf{affective images descriptions}.

\breakline

\end{document}
