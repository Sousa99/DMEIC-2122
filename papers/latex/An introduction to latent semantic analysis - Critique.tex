\documentclass{Paper_Summary}

% ================================== VARIABLES ================================== 
\renewcommand{\varpapertitle}{{An introduction to latent semantic analysis}}
\renewcommand{\varpaperauthor}{Landauer T. Foltz P. Laham D}
\renewcommand{\vardate}{{October 2021}}
% ================================== ========= ================================== 

\begin{document}
\makepapertitle

\breakline

\begin{center}
    \section*{Critique}
\end{center}
    
    \textbf{Latent Semantic Analysis (LSA)} seems to be one of the best approaches for the methodology and subject that I will be using. LSA extrapolates hidden meaning without any prior biases or knowledge as long as it has a good corpus.
 
    There is not much to comment on regarding the paper content since it simply describes the overall process and results. It was decidedly beneficial in order to understand better the overall methodology, advantages, and disadvantages. For a better description of the technique, check the \emph{Summary}.

    The search for a satisfactory \emph{Portuguese} corpus should initiate since this will be extremely helpful. The corpus should, if possible, have an informal tone since this is the one used in the interviews.

\breakline

\end{document}
